\documentclass[helvetica]{seminar} 
\input{xy}
\xyoption{all}
\usepackage{graphicx} 
\usepackage{slidesec} 
\usepackage{url}
\usepackage[normalem]{ulem}
\usepackage[usenames]{color}

\long\def\symbolfootnote[#1]#2{\begingroup%
\def\thefootnote{\fnsymbol{footnote}}\footnote[#1]{#2}\endgroup}

% to fix problems making landscape seminar pdfs
% Letter...
\pdfpagewidth=11truein
\pdfpageheight=8.5truein
\pdfhorigin=1truein     % default value(?), but doesn't work without
\pdfvorigin=1truein     % default value(?), but doesn't work without
% A4
%\pdfpagewidth=297truemm % your milage may vary....
%\pdfpageheight=210truemm
%\pdfhorigin=1truein     % default value(?), but doesn't work without
%\pdfvorigin=1truein     % default value(?), but doesn't work without



\renewcommand{\familydefault}{\sfdefault}  
 
\input{seminar.bug} 
\input{seminar.bg2} % See the Seminar bugs list 
 
\slideframe{none} 
 
 
\usepackage{fancyhdr} 
 
% Headers and footers personalization using the `fancyhdr' package 
\fancyhf{} % Clear all fields 
\renewcommand{\headrulewidth}{0mm} 
\renewcommand{\footrulewidth}{0.1mm} 
 
\fancyfoot[L]{\tiny RTCWEB Interim; Feb 2012} 
\fancyfoot[C]{\tiny Security}
\fancyfoot[R]{\tiny \theslide} 
 
 
% To center horizontally the headers and footers (see seminar.bug) 
\renewcommand{\headwidth}{\textwidth} 

% To adjust the frame length to the header and footer ones 
\autoslidemarginstrue 
\pagestyle{fancy} 
 

\newcommand{\heading}[1]{% 
  \begin{center} 
    \large\bf 
    #1 
  \end{center} 
  \vspace{.4 in}} 

\begin{document}
\begin{slide}
\begin{center}
\vspace{1 in}
\LARGE{{\bf}RTCWEB Architecture Open Issues}\\
\vspace{.2in}
\large{{Interim Meeting; February 2012}} \\
\vspace{3em}
\large{
\begin{tabular}{c}
Eric Rescorla \\
\url{ekr@rtfm.com}
\end{tabular}
}
\end{center}

\end{slide}


\centerslidesfalse

\begin{slide}
\heading{Overview}

\begin{itemize}
\item Security architecture document adopted after Taipei
  \begin{itemize}
  \item \verb^draft-ietf-rtcweb-security-arch-00^
  \end{itemize}
\item General agreement on a lot of issues
\item Purpose of the next 30 min
  \begin{itemize}
  \item Survey the open issues
  \item Resolve any that are easy
  \end{itemize}
\end{itemize}
\end{slide}


\begin{slide}
\heading{Issue: Mixed Content}

\begin{itemize}
\item Consent is granted by origin
\item What about active mixed content?
  \begin{itemize}
  \item \url{https://www.example.com/} loads script from \url{http://www.example.com}
  \item What are the PeerConnection permissions
  \end{itemize}

\item Current draft says: treat page as the HTTP origin
  \begin{itemize}
  \item Browser security experts: \textbf{``NOOOOO!!!!!!!!''}
  \end{itemize}

\end{itemize}
\end{slide}



\begin{slide}
\heading{How Browsers Handle Active Mixed Content Now}

\begin{center}
\begin{tabular}{l p{2.5in}}
Browser & Action \\
\hline
Chrome & Allow with warning -- (soon to be block) \\
Firefox & Big warning dialog \\
IE & Block \\
Safari & Accept \\
\end{tabular}
\end{center}

\end{slide}



\begin{slide}
\heading{Proposed Resolution}

\begin{itemize}
\item MUST treat HTTP and HTTPS origins as separate [uncontroversial]
\item SHOULD~\symbolfootnote[1]{Should this be a MUST?} either:
  \begin{itemize}
  \item Forbid all active mixed content [better, but out of scope]
  \item Remove RTCWEB permissions for origins with mixed content
  \end{itemize}

\item Comments?
\end{itemize}

\vspace{.5in}
\end{slide}



\begin{slide}
\heading{Issue: Consent Freshness/Keepalives}

\begin{itemize}
\item Problem: How to verify continuing consent?
  \begin{itemize}
  \item Need some sort of keepalive
  \item ICE keepalives are STUN Binding Indications (one-way)
  \end{itemize}

\item Proposal: use STUN Binding Requests instead
  \begin{itemize}
  \item MUST check no less often than every 30s
  \end{itemize}

\item Comments?
\end{itemize}

\end{slide}

\begin{slide}
\heading{Issue: Media Security Requirements}

\begin{itemize}
\item DTLS/DTLS-SRTP provides the best security
  \begin{itemize}
  \item Can detect MITM with fingerprint checks (though inconvenient)
  \item Strong authentication when used with third-party IdP
  \end{itemize}

\item Demand for SDES, RTP, or both
  \begin{itemize}
  \item Mostly in terms of interop with legacy systems w/o media gatewaying
  \item Concerns about bid-down attacks, UI confusion, etc.
  \end{itemize}
\end{itemize}
\end{slide}



\begin{slide}
\heading{Interaction with Server-based features}

\vspace{-.4in}
\begin{itemize}
\item End-to-end security requires denying the server direct media access
  \begin{itemize}
  \item Via MediaStream APIs [See Randell \sout{JSEP}Jesup's slides]
  \item This precludes many fancy video applications
  \end{itemize}

\item Need for two security models
  \begin{itemize}
  \item[] \emph{Browser-to-browser secure.} Limited effects but media protected from JS
  \item[] \emph{Javascript-visible.} Powerful effects but need to trust JS
  \item This should be under control of the JavaScript but with a UI indicator
  \end{itemize}

\item End-to-end crypto still adds value in server-visible model
  \begin{itemize}
  \item Protection against programmer error (excess logging, XSS, etc.)
  \item Malicious activity more readily detectable
  \end{itemize}

\end{itemize}
\end{slide}

\begin{slide}
\heading{Interop Deployment Questions}

\begin{itemize}
\item Everyone supports RTP
  \begin{itemize}
  \item But obviously security is... bad
  \end{itemize}

\item Most \emph{current} implementations support SDES
  \begin{itemize}
  \item Unclear (at least to me) how many deployments support it
  \end{itemize}

\item Decision proposal:
  \begin{itemize}
  \item Need RTP if not much SDES deployment
  \item If a lot of SDES deployment, not much need for RTP
  \end{itemize}


\end{itemize}

\end{slide}


\begin{slide}
\heading{Communications Security: Implementation Requirements (Proposed)}

\begin{itemize}
\item MUST implement DTLS-SRTP (for media) and DTLS (for data) 
\item MAY implement RTP(?) and SDES(?)
\item Security MUST be default state
  \begin{itemize}
  \item Implementations MUST offer DTLS and/or DTLS-SRTP for every channel
  \item MUST accept DTLS and/or DTLS-SRTP whenever offered 
  \item MUST not do unencrypted data channel
  \end{itemize}
\end{itemize}

\end{slide}



\centerslidestrue

\begin{slide}
\begin{center}
\vspace{1 in}
\LARGE{{\bf}RTCWEB Generic Identity Service}\\
\vspace{.2in}
\large{{Interim Meeting; February 2012}} \\
\vspace{3em}
\large{
\begin{tabular}{c}
Eric Rescorla \\
\url{ekr@rtfm.com}
\end{tabular}
}
\end{center}

\end{slide}


\centerslidesfalse



\begin{slide}
\heading{What are we trying to accomplish?}

\begin{itemize}
\item Allow Alice and Bob to have a secure call
  \begin{itemize}
  \item Authenticated with their identity providers
  \item On any site
    \begin{itemize}
    \item Even untrusted/partially trusted ones
    \end{itemize}
  \end{itemize}

\item Advantages
  \begin{itemize}
  \item Use one identity on any calling site
  \item Security against active attack by calling site
  \item Support for federated cases
  \end{itemize}
\end{itemize}

\end{slide}



\begin{slide}
\heading{Topology}

\begin{center}
\includegraphics[width=3in]{full-topology}
\end{center}

\end{slide}



\begin{slide}
\heading{Terminology}
\begin{itemize}
\item[] \emph{Authenticating Party (AP)}:  The entity which is trying to establish its identity.

\item[] \emph{Identity Provider (IdP)}:  The entity which is vouching for the AP's identity.

\item[] \emph{ Relying Party (RP)}:  The entity which is trying to verify the AP's identity.
\end{itemize}

\end{slide}





\begin{slide}
\heading{Types of IdP}

\begin{itemize}
\item[] \emph{Authoritative}: Attests for identities within their own namespace
  \begin{itemize}
  \item Often multiple Authoritatives IdPs exist with different scopes
  \item Examples: DNSSEC, RFC 4474, Facebook Connect (for the Facebook ID)
  \end{itemize}

\item[] \emph{Third-party}: Attests for identities in a name-space they don't control
  \begin{itemize}
  \item Often multiple Third-Party IdPs share the same space
  \item Can attest to real-world identities
  \item Examples: SSL/TLS certificates, the State of California (driver's licenses)
  \end{itemize}
\end{itemize}
\end{slide}


\begin{slide}
\heading{Authoritative vs. Third-Party IdPs: Trust Relationship}

\begin{itemize}
\item No need to explicitly trust authoritative IdPs
  \begin{itemize}
  \item \url{ekr@example.com} is whoever \url{example.com} says it is
  \item The problem is authenticating \url{example.com}
  \end{itemize}

\item Third-party IdPs need to be explicitly trusted
  \begin{itemize}
  \item Example: how do I know GoDaddy is a legitimate CA?
  \item Answer: the browser manufacturer vetted them
  \item They are allowed to attest to \emph{any} domain name
  \end{itemize}
\end{itemize}
\end{slide}


\begin{slide}
\heading{User Relationships with IdPs}

\begin{itemize}
\item Authenticating Party
  \begin{itemize}
  \item Has some account with the IdP
  \item May have established their identity 
    \begin{itemize}
    \item Especially for third-party IdPs
    \end{itemize}
  \item Can authenticate to the IdP in the future (e.g., with a password)
  \end{itemize}

\item Relying party
  \begin{itemize}
  \item Doesn't have any account relationship with the IdP\symbolfootnote[1]{Note: privacy issues.}
  \item Must be able to verify the IdP's identity
  \item Needs to trust third-party IdPs
  \end{itemize}

\end{itemize}
\end{slide}


\begin{slide}
\heading{Web-based IdP Systems}

\begin{itemize}
\item Facebook Connect
\item Google login
\item OAuth
\item OpenID
\item BrowserID
\end{itemize}

\end{slide}


\begin{slide}
\heading{Web-based IdP Objectives: User Perspective}

\begin{itemize}
\item Single-sign on
  \begin{itemize}
  \item No need to make a new account for each service
  \item Don't need to remember lots of passwords
  \end{itemize}

\item Privacy
  \begin{itemize}
  \item Avoid creating a super-cookie
    \begin{itemize}
    \item Only authenticate to sites I have approved
    \item Control exposure of my personal information
    \end{itemize}
  \end{itemize}
\end{itemize}
\end{slide}


\begin{slide}
\heading{Web-based IdP Objectives: Site Perspective}

\begin{itemize}
\item Low friction
  \begin{itemize}
  \item Avoid the need for account creation
  \item ... the source of a lot of user rolloff
  \end{itemize}

\item Leverage existing user information
  \begin{itemize}
  \item E.g., information you've stored in your FB account
  \end{itemize}
\end{itemize}
\end{slide}


\begin{slide}
\heading{Example: Facebook Connect (sorta OAuth)}

\begin{itemize}
\item AP is a user with a Facebook account
  \begin{itemize}
  \item They may or may not be logged in at the moment
  \item (Where \emph{logged in} == cookies)
  \end{itemize}

\item RP is a Web server
  \begin{itemize}
  \item Idea is to bootstrap Facebook authentication
  \item ... rather than have your own account system
  \item RP registers with Facebook and gets an application key
    \begin{itemize}
    \item Facebook wants to control authentication experience
    \end{itemize}
  \end{itemize}
\end{itemize}
\end{slide}



\begin{slide}
\heading{Facebook Connect Call Flow (not logged in) 1}

\vspace{-.7in}
$$
\xymatrix@C=1.0in@R=.15in{
  \txt{Alice} & \txt{RP\\\url{www.example.com}} & \txt{Facebook} \\
  \ar[r]^{\txt{\texttt{GET /...}}} & & \\
  & \ar[l]_{\txt{Redirect to}}^{\txt\tiny{\texttt{www.facebook.com/dialog/oauth?client\_id=1234\&redirect\_uri=www.example.com/auth}}} & \\
  \ar[rr]^{\txt\tiny{\texttt{GET /dialog/oauth?client\_id=1234\&redirect\_uri=www.example.com/auth}}} & & \\
  & & \ar@{<->}[ll]_{\txt{Login and permissions dialog}} \\
}
$$

\end{slide}



\begin{slide}

\begin{center}
\includegraphics[width=3.5in]{fb-permissions}
\end{center}

\end{slide}

\begin{slide}
\heading{Facebook Connect Call Flow (not logged in) 2}

\vspace{-.7in}
$$
\xymatrix@C=1.0in@R=.15in{
  \txt{Alice} & \txt{RP\\\url{www.example.com}} & \txt{Facebook} \\
  \ar[r]^{\txt{\texttt{GET /...}}} & & \\
  & \ar[l]_{\txt{Redirect to}}^{\txt\tiny{\texttt{www.facebook.com/dialog/oauth?client\_id=1234\&redirect\_uri=www.example.com/auth}}} & \\
  \ar[rr]^{\txt\tiny{\texttt{GET /dialog/oauth?client\_id=1234\&redirect\_uri=www.example.com/auth}}} & & \\
  & & \ar@{<->}[ll]_{\txt{Login and permissions dialog}} \\
  & & \ar[ll]_{\txt{Redirect to}}^{\txt\tiny{\texttt{www.example.com/auth?code=5678}}}\\
  \ar[r]^{\txt\tiny{\texttt{GET /auth?code=5678}}} & &\\
  & \ar[r]^{\txt\tiny{\texttt{GET /oauth/access\_token?client\_id=1234\&client\_secret=<secret>\&code=5678}}} & \\
  & & \ar[l]_{\txt\tiny{\texttt{access\_token=987654321}}} \\
  & \ar[r]^{\txt\tiny{\texttt{GET /me?access\_token=987654321}}} & \\
  & & \ar[l]_{\txt\tiny{\texttt{user=1111111, ...}}} \\
  & \ar[l]_{\txt{Hello, user 1111111}} & \\
}
$$

\end{slide}




\begin{slide}
\heading{Facebook Connect Call Flow (logged in)}

\vspace{-.7in}
$$
\xymatrix@C=1.0in@R=.15in{
  \txt{Alice} & \txt{RP\\\url{www.example.com}} & \txt{Facebook} \\
  \ar[r]^{\txt{\texttt{GET /...}}} & & \\
  & \ar[l]_{\txt{Redirect to}}^{\txt\tiny{\texttt{www.facebook.com/dialog/oauth?client\_id=1234\&redirect\_uri=www.example.com/auth}}} & \\
  \ar[rr]^{\txt\tiny{\texttt{GET /dialog/oauth?client\_id=1234\&redirect\_uri=www.example.com/auth}}} & & \\
  & & \ar[ll]_{\txt{Redirect to}}^{\txt\tiny{\texttt{www.example.com/auth?code=5678}}}\\
  \ar[r]^{\txt\tiny{\texttt{GET /auth?code=5678}}} & &\\
  & \ar[r]^{\txt\tiny{\texttt{GET /oauth/access\_token?client\_id=1234\&client\_secret=<secret>\&code=5678}}} & \\
  & & \ar[l]_{\txt\tiny{\texttt{access\_token=987654321}}} \\
  & \ar[r]^{\txt\tiny{\texttt{GET /me?access\_token=987654321}}} & \\
  & & \ar[l]_{\txt\tiny{\texttt{user=1111111, ...}}} \\
  & \ar[l]_{\txt{Hello, user 1111111}} & \\
}
$$

\end{slide}


\begin{slide}
\heading{Facebook Connect Privacy Features}

\begin{itemize}
\item RP needs to register with Facebook
\item User approves policy separately for each RP
  \begin{itemize}
  \item Including which user information to share
  \end{itemize}

\item Facebook learns about every authentication transaction
  \begin{itemize}
  \item Including user/RP pair
  \end{itemize}

\end{itemize}

\end{slide}



\begin{slide}
\heading{Example: BrowserID}

\begin{itemize}
\item Effectively client-side certificates
  \begin{itemize}
  \item But user not exposed to certificates
  \end{itemize}

\item Why this example?
  \begin{itemize}
  \item Easy to understand 
  \item Familiar-looking technology
  \item Less need to wrap your head around redirects, etc.
  \end{itemize}


\end{itemize}

\end{slide}

\begin{slide}
\heading{BrowserID (no key pair)}

\vspace{-.5in}
$$
\xymatrix@C=0.8in@R=.15in{
  \txt{Alice} & \txt{RP\\\url{www.example.com}} & \txt{BrowserID.org} \\
    \ar[r]^{\txt{\texttt{GET /...}}} & & \\
    & \ar[l]_{\txt\tiny{\texttt{<script src="https://browserid.org/include.js"/>}}}^{\txt\tiny{\texttt{navigator.id.get(function(assertion)\{...\});}}} & \\
    \txt{[Generate Keys]} \\
    \ar[rr]^{\txt{Get certificate + Cookie}} & & \\
    & & \ar[ll]_{\txt{Certificate}} \\
    \txt{[Sign Assertion]} \\
    \ar[r]^{\txt{Signed assertion + Certificate}} & \\
    & \ar[l]_{\txt{Hello, user 11111111}} \\
}
$$

\end{slide}

\begin{slide}

\includegraphics[width=4in]{browserid-login.png}

\end{slide}





\begin{slide}
\heading{BrowserID: Why no MITM Attacks?}

\vspace{-.5in}
$$
\xymatrix@C=0.7in@R=.15in{
  \txt{Alice} & \txt{\url{attacker.com}} & \txt{\url{example.com}} \\
    \ar[r]^{\txt{\texttt{GET /...}}} & & \\
    & \ar[r]^{\txt{\texttt{GET /...}}} & \\
    & \ar[l]_{\txt\tiny{\texttt{<script src="https://browserid.org/include.js"/>}}}^{\txt\tiny{\texttt{navigator.id.get(function(assertion)\{...\});}}} & \\
    \txt{[Sign Assertion]} \\
    \ar[r]^{\txt{Signed assertion + Certificate}} & \\
    &   \ar[r]^{\txt{Signed assertion + Certificate}} & \\
    & & \ar[l]_{\txt{Hello, user 11111111}} \\
}
$$
\end{slide}

\begin{slide}
\heading{BrowserID: Audience Parameter}

\vspace{-.5in}
$$
\xymatrix@C=0.7in@R=.15in{
  \txt{Alice} & \txt{\url{attacker.com}} & \txt{\url{example.com}} \\
    \ar[r]^{\txt{\texttt{GET /...}}} & & \\
    & \ar[r]^{\txt{\texttt{GET /...}}} & \\
    & \ar[l]_{\txt\tiny{\texttt{<script src="https://browserid.org/include.js"/>}}}^{\txt\tiny{\texttt{navigator.id.get(function(assertion)\{...\});}}} & \\
    \txt{[Sign Assertion]} \\
    \ar[r]^{\txt{Signed assertion({\color{blue} audience=\texttt{attacker.com}}) + Certificate}} & \\
    &   \ar[r]^{\txt{Signed assertion + Certificate}} & \\
    & & \ar[l]_{\txt{Audience mismatch error}} \\
}
$$
\end{slide}

\begin{slide}
\heading{Preventing assertion forwarding}

\begin{itemize}
\item BrowserID assertions are scoped to origin (audience parameter)
  \begin{itemize}
  \item RPs check that the origin in the assertion matches their domain
  \item This prevents assertion forwarding
  \end{itemize}

\item Why does this work?
  \begin{itemize}
  \item BrowserID JS is part of the TCB
  \item Browser enforces origin of requests from the calling site
  \item RP transitively trusts origin/audience because it trusts BrowserID.org
  \end{itemize}
\end{itemize}

\end{slide}
        



\begin{slide}
\heading{Browser-ID Privacy Features}

\begin{itemize}
\item Client generates a key pair
  \begin{itemize}
  \item Idp signs a binding between key pair and user ID
  \end{itemize}

\item Client generates assertions based on key pair
  \begin{itemize}
  \item Sends along certificate
  \end{itemize}

\item RP fetches IdP public key
  \begin{itemize}
  \item This need only happen once
  \end{itemize}

\item IdP never learns where you are visiting
  \begin{itemize}
  \item No relationship between RP and IdP
  \end{itemize}


\end{itemize}

\end{slide}


\begin{slide}
\heading{Example: BrowserID (existing key pair)}

\vspace{-.5in}
$$
\xymatrix@C=0.8in@R=.15in{
  \txt{Alice} & \txt{RP\\\url{www.example.com}} & \txt{BrowserID.org} \\
    \ar[r]^{\txt{\texttt{GET /...}}} & & \\
    & \ar[l]_{\txt\tiny{\texttt{<script src="https://browserid.org/include.js"/>}}}^{\txt\tiny{\texttt{navigator.id.get(function(assertion)\{...\});}}} & \\
    \txt{[Sign Assertion]} \\
    \ar[r]^{\txt{Signed assertion + Certificate}} & \\
    & \ar[l]_{\txt{Hello, user 11111111}} \\
}
$$

\end{slide}





\begin{slide}
\heading{BrowserID Security Architecture}

\begin{center}
\includegraphics[width=3.2in]{browserid-arch}
\end{center}

\end{slide}


\begin{slide}
\heading{One browser, multiple security contexts}

\begin{itemize}
\item Browser security data scoped by \emph{origin}
  \begin{itemize}
  \item \url{browserid.org} window and \url{myfavoritebeer.org} window are isolated
  \item Each runs their own JS independently
  \end{itemize}

\item Security guarantees
  \begin{itemize}
  \item Origin A can't touch origin B's data
  \item Origin A can't see what origin B is displaying
  \item Communication is by \verb^postMessage()^ (or navigation hack)
  \end{itemize}
\end{itemize}
\end{slide}


\begin{slide}
\heading{PostMessage: Sender}

\verb^otherWindow.postMessage(message, targetOrigin);^

\begin{itemize}
\item[] \verb^otherWindow^: the window to send the message to
\item[] \verb^message^: the message to send
\item[] \verb^targetOrigin^: the expected origin of the other window
\end{itemize}
\end{slide}


\begin{slide}
\heading{Why do we need \texttt{targetOrigin}?}

\begin{itemize}
\item Malicious pages can navigate other windows
  \begin{itemize}
  \item This creates a race condition
  \end{itemize}

\item RP creates the new window to IdP with \verb^w = createWindow()^
\item Attacker navigates \verb^w^ to his own site
\item RP does \verb^w.postMessage(secret,...)^
\item Attacker gets the secret
\item \verb^targetOrigin^ stops this
\end{itemize}

\end{slide}


\begin{slide}
\heading{PostMessage: receiver}

\begin{verbatim}
window.addEventListener('message',
                        function(event) {
                          ...
                        });
\end{verbatim}

\begin{itemize}
\item Event properties:
  \begin{itemize}
  \item[] \verb^data^: the message passed by the sender
  \item[] \verb^origin^: the sender's origin
  \item[] \verb^source^: the sender's window
  \end{itemize}

\item Important: \verb^origin^ value can be trusted
  \begin{itemize}
  \item Enforced by the browser
  \item May not be the current origin of \verb^source^, however
  \end{itemize}
\end{itemize}
\end{slide}


\begin{slide}
\heading{IFRAMEs}

\begin{itemize}
\item What if I don't want another window to open?
  \begin{itemize}
  \item Solution: IFRAMEs
  \end{itemize}

  \begin{center}
  \includegraphics[height=2in]{iframe}
  \end{center}

\end{itemize}
\end{slide}


\begin{slide}
\heading{IFRAME Security Properties}

\begin{itemize}
\item Isolated from the main page
  \begin{itemize}
  \item More or less the same rules as a separate window
  \end{itemize}

\item Can be easily navigated by the main page
\item Can be invisible (both good and bad)
\end{itemize}
\end{slide}



\begin{slide}

\heading{Logins generally done in separate windows}

\vspace{-.4in}
\begin{center}
\includegraphics[width=2.5in]{fb-login}
\end{center}

\end{slide}



\begin{slide}
\heading{Why aren't logins done in IFRAMEs?}

\begin{itemize}
\item Scenario: you are on \url{example.org}
  \begin{itemize}
  \item \url{example.org} wants to log you in with \url{idp.org}
  \end{itemize}

\item Both Facebook Connect and BrowserID use a separate window
\item Why?
  \begin{itemize}
  \item IdP is soliciting the user's password
  \item User needs to know they are using the right IdP
  \item A separate window means they can examine the URL bar
  \item Also concerns about clickjacking/redressing
  \end{itemize}

\item Other option is to navigate the entire page to an interstitial page
\end{itemize}
\end{slide}



\begin{slide}
\heading{How Clickjacking Works}

\begin{itemize}
\item Attacker embeds the victim site's page in an IFRAME
  \begin{itemize}
  \item IFRAME is in front but marked transparent
  \item The attacker's page shows through
  \end{itemize}

\item Attacker gets the victim to click on ``his'' page
  \begin{itemize}
  \item Really the victim site's page
  \end{itemize}

\item Victim has just taken action on the victim site
\end{itemize}

\end{slide}


\begin{slide}
\heading{IFRAMEs, Clickjacking, and Permissions Grants}

\begin{center}
\includegraphics[width=3.5in]{clickjacking}
\end{center}

\end{slide}


\begin{slide}
\heading{Preventing Framing}

\begin{itemize}
\item IdP policy is to have the login page be top-level
  \begin{itemize}
  \item Good RPs comply with this policy
  \item But we're concerned about malicious RPs
  \end{itemize}

\item IdPs use ``framebusting'' JavaScript to prevent being framed
  \begin{itemize}
  \item This is harder than it sounds
  \item ... but standard procedure
  \end{itemize}


\end{itemize}

\end{slide}


\begin{slide}
\heading{IFRAMEs don't have to be visible}

\begin{verbatim}
	idp = document.createElement('IFRAME');
	$(idp).hide();

\end{verbatim}

\begin{itemize}
\item This takes up no space on the screen
\begin{itemize}
\item It's just JS from the IFRAME source running on the page
\item Can still \verb^postMessage()^ to and from it
\end{itemize}
\item Invisible IFRAMEs are a very important tool
\end{itemize}
\end{slide}



\begin{slide}
\heading{What are we trying to accomplish?}

\vspace{-.3in}
\begin{itemize}
\item Repurpose existing identity infrastructure for user-to-user authentication
\item []
\item Requirements/objectives
  \begin{itemize}
  \item Use existing accounts
  \item Minimal (preferably no) changes to IdP
  \item Easy to support at calling site
    \begin{itemize}
    \item Better if no change
    \end{itemize}

  \item Generic support in browser
    \begin{itemize}
    \item Single downward interface between PeerConnection object and IdP
    \item Should be able to support new IdPs/protocols without changing browser
    \end{itemize}
  \end{itemize}

\end{itemize}

\end{slide}

\begin{slide}
\heading{Reminder: Trust Architecture}

\begin{center}
\includegraphics[width=3.5in]{rtcweb-security-arch-no-idp}
\end{center}

\end{slide}




\begin{slide}
\heading{Example IdP Interaction: BrowserID}

\includegraphics[width=3.5in]{browser-id-arch}
\end{slide}

\begin{slide}
\heading{Example ROAP Offer with BrowserID}

\begin{tiny}
\begin{verbatim}
      {
        "messageType":"OFFER",
        "callerSessionId":"13456789ABCDEF",
        "seq": 1
        "sdp":"
        ...
      a=fingerprint: SHA-1 \
      4A:AD:B9:B1:3F:82:18:3B:54:02:12:DF:3E:5D:49:6B:19:E5:7C:AB\n",
       "identity":{
            "idp":{     // Standardized
               "domain":"browserid.org",
               "method":"default"
            },
            "assertion":   // Contents are browserid-specific
              "\"assertion\": {
                \"digest\":\"<hash of the contents from the browser>\",
                \"audience\": \"[TBD]\"
                \"valid-until\": 1308859352261,
               },
               \"certificate\": {
                 \"email\": \"rescorla@example.org\",
                 \"public-key\": \"<ekrs-public-key>\",
                 \"valid-until\": 1308860561861,
               }" // certificate is signed by example.org
            }
      }
\end{verbatim}
\end{tiny}
\end{slide}



\begin{slide}
\heading{Example JSEP TransportInfo with Facebook Connect\\(Or any private identity service)}

\vspace{-.3in}
\begin{tiny}
\begin{verbatim}
{
       "pwd":"asd88fgpdd777uzjYhagZg",
       "ufrag":"8hhy",
       "fingerprint":{
          "algorithm":"sha-1",
          "value":"4AADB9B13F82183B540212DF3E5D496B19E57CAB",
       },
       "candidates:[
         ...
       ],
       "identity":{
         "idp":{
            "domain": "example.org"
            "protocol": "bogus"
          },
          "assertion":\"{\"identity\":\"bob@example.org\",
                        \"contents\":\"abcdefghijklmnopqrstuvwyz\",
                        \"signature\":\"010203040506\"}"
       }
}
\end{verbatim}
\end{tiny}                      

\begin{itemize}
\item[] * Assumption here is that we have changed JSEP to emit transport-infos
\end{itemize}

\end{slide}



\begin{slide}
\heading{But we want it to be generic...}

\begin{itemize}
\item This means defined interfaces
\item ... that work for any IdP
\end{itemize}

\end{slide}


\begin{slide}
\heading{What needs to be defined}

\begin{itemize}
\item Information from the signaling message that is authenticated [IETF]
  \begin{itemize}
  \item Minimally: DTLS-SRTP fingerprint
  \item Generic carrier for identity assertion
  \item Depends on signaling protocol
  \end{itemize}

\item Interface from PeerConnection to the IdP [IETF]
  \begin{itemize}
  \item A specific set of messages to exchange
  \item Sent via \verb^postMessage()^ or WebIntents
  \end{itemize}

\item JavaScript calling interfaces to PeerConnection [W3C]
  \begin{itemize}
  \item Specify the IdP
  \item Interrogate the connection identity information
  \end{itemize}
\end{itemize}
\end{slide}


\begin{slide}
\heading{What needs to be tied to user identity?}

\begin{itemize}
\item Only data which is verifiably bound is trustworthy
  \begin{itemize}
  \item Need to assume attacker has modified anything else
  \end{itemize}

\item Initial analysis (depends on protocol)
  \begin{itemize}
  \item Fingerprint (MUST)
  \item ICE candidates
  \item Media parameters
  \end{itemize}
\end{itemize}
\end{slide}



\begin{slide}
\heading{Security Properties of ICE Candidates}

\begin{itemize}
\item Effect of modifying ICE candidates
  \begin{itemize}
  \item Advertise candidates to route media through attacker
    \begin{itemize}
    \item Makes a MITM attack easier
    \item Mostly irrelevant if DTLS keying used
    \end{itemize}
  \item Route to \verb^/dev/null^ (DoS)
    \begin{itemize}
    \item Silly if you are in signaling path!
    \end{itemize}

  \end{itemize}

\item Signaling service can affect ICE candidates anyway
  \begin{itemize}
  \item Provide a malicious TURN server
  \item Return blackhole server reflexive addresses
  \item This drives data through signaling service
  \end{itemize}
\end{itemize}
\end{slide}



\begin{slide}
\heading{Security Properties of Media Parameters}

\begin{itemize}
\item Which media flows
  \begin{itemize}
  \item Calling service has control of this anyway
  \item But the UI needs to show what is being used
    \begin{itemize}
    \item For consent reasons
    \end{itemize}
  \end{itemize}

\item Which codecs
  \begin{itemize}
  \item Calling service can influence these
  \item Might be nice to secure them
  \item But too limiting
  \item SRTP should provide security regardless of codec selection
  \end{itemize}

\end{itemize}

\end{slide}




\begin{slide}
\heading{Generic Structure for Identity Assertions}

\begin{small}
\begin{verbatim}
       "identity":{
            "idp":{     // Standardized
               "domain":"idp.example.org", // Identity domain
               "method":"default"          // Domain-specific method
            },
            "assertion": "..."             // IdP-specific
       }
\end{verbatim}
\end{small}
\end{slide}


\begin{slide}
\heading{Basic Architecture}

\end{slide}


\begin{slide}
\heading{IdP Trust Architecture: Authenticating Party}

\begin{center}
\includegraphics[width=3.5in]{rtcweb-security-arch-idp}
\end{center}

\end{slide}

\begin{slide}
\heading{IdP Trust Architecture: Relying Party}

\begin{center}
\includegraphics[width=3.5in]{rtcweb-security-arch-idp2}
\end{center}
\end{slide}



\begin{slide}
\heading{Generic Downward Interface\\(Implemented by PeerConnection)}

\begin{itemize}
\item Instantiate ``IdP Proxy'' with JS from IdP
  \begin{itemize}
  \item Probably invisible IFRAME
  \item Maybe a WebIntent (more later)
  \end{itemize}

\item Send (standardized) messages to IdP proxy via \verb^postMessage()^
  \begin{itemize}
  \item ``\verb^SIGN^'' to get assertion
  \item ``\verb^VERIFY^'' to verify assertion
  \end{itemize}

\item IdP proxy responds 
  \begin{itemize}
  \item ``\verb^SUCCESS^'' with answer
  \item ``\verb^ERROR^'' with error
  \end{itemize}
\end{itemize}

\end{slide}



\begin{slide}
\heading{Where is the IdP JS fetched from?}

\begin{itemize}
\item Deterministically constructed from IdP domain name and method
  \begin{itemize}
  \item[] {\small \url{https://<idp-domain>/.well-known/idp-proxy/<protocol>}}
    \end{itemize}

\item Why in \verb^/.well-known^?
  \begin{itemize}
  \item Trust-relationship derives from control of the domain
  \item Must not be possible for non-administrative users of domain to impersonate IdP
  \end{itemize}
\end{itemize}

\end{slide}

\begin{slide}
\heading{How does PeerConnection know IdP domain?}

\begin{itemize}
\item Authenticating Party

  \begin{itemize}
  \item IdP domain configured into browser
    \begin{itemize}
    \item User ``logs into'' browser via UI
    \item WebIntents again
    \end{itemize}

  \item Specified by the calling site
    \begin{itemize}
    \item ``Authenticate this call with Facebook connect''
    \item Need a new API point for this
    \end{itemize}
  \end{itemize}

\item Relying party
  \begin{itemize}
  \item Carried in the generic part of the identity assertion
  \end{itemize}
\end{itemize}

\end{slide}



\begin{slide}
\heading{Generic Message Structure}

\begin{verbatim}
    {
      "type": "...",     // "SIGN","VERIFY","SUCCESS", ...
      "id": "1",         // used for correlation
    }
\end{verbatim}

\end{slide}


\begin{slide}
\heading{Incoming Message Checks (IdP Proxy)}

\begin{itemize}
\item Messages MUST come from \verb^rtcweb://.../^
\item This prevents ordinary JS from instantiating IdP proxy
  \begin{itemize}
  \item Remember, it's just an IFRAME
  \item But you can't set your origin to arbitrary values
  \end{itemize}

\item Messages MUST come from parent window
  \begin{itemize}
  \item Prevents confusion about which proxy
  \end{itemize}
\end{itemize}
\end{slide}



\begin{slide}
\heading{Incoming Message Checks (PeerConnection)}

\begin{itemize}
\item Messages MUST come from IdP origin domain
  \begin{itemize}
  \item Prevents navigation by attackers in other windows
  \end{itemize}

\item Messages MUST come from IdP proxy window
  \begin{itemize}
  \item Prevents confusion about which proxy
  \end{itemize}
\end{itemize}

\end{slide}


\begin{slide}
\heading{Signature process}

\begin{tiny}
\begin{verbatim}
       PeerConnection -> IdP proxy:
         {
           "type":"SIGN",
            "id":1,
            "message":"abcdefghijklmnopqrstuvwyz"
         }

       IdPProxy -> PeerConnection:
         {
           "type":"SUCCESS",
           "id":1,
           "message": {
             "idp":{
               "domain": "example.org"
               "protocol": "bogus"
             },
             "assertion":\"{\"identity\":\"bob@example.org\",
                            \"contents\":\"abcdefghijklmnopqrstuvwyz\",
                            \"signature\":\"010203040506\"}"
           }
         }

\end{verbatim}
\end{tiny}
\end{slide}



\begin{slide}
\heading{Verification Process}

\begin{tiny}
\begin{verbatim}
         PeerConnection -> IdP Proxy:
           {
             "type":"VERIFY",
             "id":2,
             "message":\"{\"identity\":\"bob@example.org\",
                          \"contents\":\"abcdefghijklmnopqrstuvwyz\",
                          \"signature\":\"010203040506\"}"
           }

         IdP Proxy -> PeerConnection:
           {
            "type":"SUCCESS",
            "id":2,
            "message": {
              "identity" : {
                "name" : "bob@example.org",
                "displayname" : "Bob"
              },
              "contents":"abcdefghijklmnopqrstuvwyz"
            }
           }
\end{verbatim}
\end{tiny}

\end{slide}


\begin{slide}
\heading{Meaning of Successful Verification}

\begin{itemize}
\item IdP has verified assertion
  \begin{itemize}
  \item Identity is given in ``identity''
  \item ``name'' is the actual identity (RFC822 format)
  \item ``displayname'' is a human-readable string
  \end{itemize}

\item Contents is the original message the AP passed in
\end{itemize}

\end{slide}


\begin{slide}
\heading{Processing Successful Verifications}

\begin{itemize}
\item Authoritative IdPs
  \begin{itemize}
  \item RHS of \verb^identity.name^ matches IdP domain
  \item No more checks needed
  \end{itemize}

\item Third-party IdPs
  \begin{itemize}
  \item RHS of \verb^identity.name^ does not match IdP domain
  \item IdP MUST be trusted by policy
  \end{itemize}

\item These checks performed by PeerConnection
\end{itemize}

\end{slide}

\begin{slide}
\heading{How do I stand up a new IdP?}

\begin{enumerate}
\item Get some users (the hard part)
\item Implement handlers for \verb^SIGN^ and \verb^VERIFY^ messages
  \begin{itemize}
  \item Probably $<100$ lines of JS
  \end{itemize}

\item Put the right JS at \verb^/.well-known/idp-proxy^
\item Profit
\end{enumerate}
\end{slide}


\begin{slide}
\heading{Integrated IdP Support}

\begin{itemize}
\item Things work fine with no browser-side IdP support
\item But specialized support is nice too
  \begin{itemize}
  \item ``Sign-in to browser'' in Chrome
  \item BrowserID in Firefox
  \item Better UI/performance properties
  \end{itemize}

\item Still specify IdP by URL
  \begin{itemize}
  \item IdP JS detects that the browser has built-in support
  \item Calls go directly to the browser code
  \end{itemize}
\end{itemize}

\end{slide}



\begin{slide}
\heading{Questions?}

\end{slide}




\end{document}
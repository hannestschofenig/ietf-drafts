\documentclass[helvetica]{seminar} 
\input{xy}
\xyoption{all}
\usepackage{graphicx} 
\usepackage{slidesec} 
\usepackage{url}
\usepackage[framemethod=TikZ]{mdframed}
\usepackage{color}
\usepackage[normalem]{ulem}  

\def\dash---{\unskip\kern.16667em---\penalty\exhyphenpenalty\hskip.16667em\ignorespaces}
\long\def\symbolfootnote[#1]#2{\begingroup%
\def\thefootnote{\fnsymbol{footnote}}\footnote[#1]{#2}\endgroup}

% to fix problems making landscape seminar pdfs
% Letter...
\pdfpagewidth=11truein
\pdfpageheight=8.5truein
\pdfhorigin=1truein     % default value(?), but doesn't work without
\pdfvorigin=1truein     % default value(?), but doesn't work without
% A4
%\pdfpagewidth=297truemm % your milage may vary....
%\pdfpageheight=210truemm
%\pdfhorigin=1truein     % default value(?), but doesn't work without
%\pdfvorigin=1truein     % default value(?), but doesn't work without


\renewcommand{\familydefault}{\sfdefault}  
 
\input{seminar.bug} 
\input{seminar.bg2} % See the Seminar bugs list 
 
\slideframe{none} 
 
 
\usepackage{fancyhdr} 
 
% Headers and footers personalization using the `fancyhdr' package 
\fancyhf{} % Clear all fields 
\renewcommand{\headrulewidth}{0mm} 
\renewcommand{\footrulewidth}{0.1mm} 
 
\fancyfoot[L]{\tiny IETF 98} 
\fancyfoot[C]{\tiny TLS}
\fancyfoot[R]{\tiny \theslide} 
 
 
% To center horizontally the headers and footers (see seminar.bug) 
\renewcommand{\headwidth}{\textwidth} 

% To adjust the frame length to the header and footer ones 
\autoslidemarginstrue 
\pagestyle{fancy} 
 

\newcommand{\heading}[1]{% 
  \begin{center} 
    \large\bf 
    #1 
  \end{center} 
  \vspace{.4 in}} 



\begin{document}

\begin{slide}
\begin{center}
\vspace{.5 in}
\LARGE{{\bf}TLS 1.3\\{\small \verb^draft-ietf-tls-tls13-19^}}\\
\vspace{.2in}
\large{
\begin{tabular}{c}
Eric Rescorla\\
Mozilla\\
\url{ekr@rtfm.com}
\end{tabular}
}
\end{center}
\end{slide}

\centerslidesfalse 

\begin{slide}
\heading{Agenda}

\begin{itemize}
\item Status
\item WGLC issues
\item Timeline
\end{itemize}

\end{slide}


\begin{slide}
\heading{Status}

\begin{itemize}
\item In WGLC\#2 with: draft-ietf-tls-tls13-19
  \begin{itemize}
  \item Modest changes from -18 (more later)
  \end{itemize}
\item Quite a few interoperable implementations
  \begin{itemize}
  \item draft-18 in Firefox Beta (NSS), Chrome Beta (BoringSSL), Cloudflare, OpenSSL, Facebook (Fizz), Fizz, OpenSSL
  \item draft-19 under development with partial interop
  \end{itemize}
\end{itemize}
\end{slide}

\begin{slide}
\heading{Additional Derive-Secret stage to key schedule}

{\scriptsize
\begin{verbatim}
                ...
                 |
                 v
           Derive-Secret(., "derived secret", "")
                 |
                 v
(EC)DHE -> HKDF-Extract = Handshake Secret
                 |
                 +-----> Derive-Secret(., "client handshake traffic secret",
                 |                     ClientHello...ServerHello)
                 |                     = client_handshake_traffic_secret
                ...
\end{verbatim}
}

\begin{itemize}
\item Added before each HKDF-Extract from a non-0 salt
\item Restore extract/expand parity
\item Prevent theoretical concern about collisions from chosen ``IKM'' values
\end{itemize}
\end{slide}

\begin{slide}
\heading{Hash the context value in exporters}

\begin{itemize}
\item The context value is limited to 255 bytes
\item But the context length in 5705 is 16 bits
\item Consensus: hash the value before feeding to HKDF
\end{itemize}
\end{slide}


\begin{slide}
\heading{Hash \texttt{ClientHello1} in transcript when doing HRR}

\begin{itemize}
\item This makes stateless HRR easier
\item Also insert the selected cipher suite in HRR
\end{itemize}

\end{slide}


\begin{slide}
\heading{Add an additional Derive-Secret stage to exporters}

\begin{itemize}
\item The EKM can be used to compute any exported value
  \begin{itemize}
  \item This means if you need a long-term exporter the EKM is a threat to other exported value
  \end{itemize}
\item Solution: domain separate exporters on \verb^label^
\end{itemize}

{\small
\begin{verbatim}
    HKDF-Expand-Label(Derive-Secret(Secret, label, ""),
                      "exporter", Hash(context_value), key_length)
\end{verbatim}
}
\end{slide}


\begin{slide}
\heading{Change \texttt{end\_of\_early\_data} to be a handshake message}

\begin{itemize}
\item It was goofy to have it an alert
\item All other state transitions are handshake messages
\item Spec isn't very clear on how this fits into the transcript
  \begin{itemize}
  \item Consensus answer: \verb^ServerFinished, EOED, [Client Certificate]...^
  \item -20 will be clearer
  \end{itemize}
\end{itemize}
\end{slide}


\begin{slide}
\heading{PR\#768: DH Key Reuse Considerations}

\begin{itemize}
\item Not that confident of the analysis
\item We don't really want to encourage re-use
\item[]
\item Proposed resolution: drop
\end{itemize}
\end{slide}


\begin{slide}
\heading{PR\#762: Short Headers}

\begin{itemize}
\item Concerns about interop
  \begin{itemize}
  \item Already seeing some interop problems without this
  \item Controlled experiments not very encouraging
  \end{itemize}
\item[]
\item Proposed resolution: drop
\end{itemize}
\end{slide}


\begin{slide}
\heading{Non-X.509 Certificates}

\begin{itemize}
\item We've changed \texttt{Certificate} a lot
\item The other certificate format documents assume you replace all
  of \verb^Certificate^, which doesn't work
\item Proposed resolution:
  \begin{itemize}
  \item Deprecate the following for TLS 1.3: \verb^{client,server}_certificate_type^, \verb^user_mapping^, \verb^cert_type^, \verb^cached_info^?
  \item People can update drafts with new code points if they want
  \end{itemize}
\end{itemize}

\end{slide}


\begin{slide}
\heading{Opting-out of post-handshake client auth}

\vspace{-2ex}
\begin{itemize}
\item Olivier Levillain on-list:
\end{itemize}

{\footnotesize
\begin{quote}
The client can not indeed ignore all this state to answer, since it
is supposed to answer at least with a Finished message, which will cover
the CertificateRequest message. Moreover, since each of these Finished
messages must cover the initial handshake and the current
CertificateRequest message, it requires a forkable hash implementation,
which requires more memory.
\end{quote}
}

\begin{itemize}
\item Potential options:
  \begin{itemize}
  \item \sout{Remove post-handshake auth}
  \item Require an extension to opt-in to post-handshake auth
  \item Specifically allow ignoring post-handshake
  \item Do nothing
  \end{itemize}

\item Proposal: ???
\end{itemize}
\end{slide}


\begin{slide}
\heading{Any other issues?}

\center{
On to draft-20 and IETF-LC  
}
\end{slide}

\end{document} 



% - Hash ClientHello1 in the transcript when HRR is used. This
%   reduces the state that needs to be carried in cookies. (*)
% 
% - Restructure CertificateRequest to have the selectors
%   in extensions. This also allowed defining a "certificate_authorities"
%   extension which can be used by the client instead of trusted_ca_keys (*).
% 
% - Tighten record framing requirements and require checking of them (*).
% 
% - Consolidate "ticket_early_data_info" and "early_data" into a single
%   extension (*).
% 
% - Change end_of_early_data to be a handshake message (*).
% 
% - Add pre-extract Derive-Secret stages to key schedule (*).

\documentclass[helvetica]{seminar} 
\input{xy}
\xyoption{all}
\usepackage{graphicx} 
\usepackage{slidesec} 
\usepackage{url}
\usepackage[framemethod=TikZ]{mdframed}
\usepackage{color}

\long\def\symbolfootnote[#1]#2{\begingroup%
\def\thefootnote{\fnsymbol{footnote}}\footnote[#1]{#2}\endgroup}

% to fix problems making landscape seminar pdfs
% Letter...
\pdfpagewidth=11truein
\pdfpageheight=8.5truein
\pdfhorigin=1truein     % default value(?), but doesn't work without
\pdfvorigin=1truein     % default value(?), but doesn't work without
% A4
%\pdfpagewidth=297truemm % your milage may vary....
%\pdfpageheight=210truemm
%\pdfhorigin=1truein     % default value(?), but doesn't work without
%\pdfvorigin=1truein     % default value(?), but doesn't work without


\renewcommand{\familydefault}{\sfdefault}  
 
\input{seminar.bug} 
\input{seminar.bg2} % See the Seminar bugs list 
 
\slideframe{none} 
 
 
\usepackage{fancyhdr} 
 
% Headers and footers personalization using the `fancyhdr' package 
\fancyhf{} % Clear all fields 
\renewcommand{\headrulewidth}{0mm} 
\renewcommand{\footrulewidth}{0.1mm} 
 
\fancyfoot[L]{\tiny IETF 92} 
\fancyfoot[C]{\tiny TLS}
\fancyfoot[R]{\tiny \theslide} 
 
 
% To center horizontally the headers and footers (see seminar.bug) 
\renewcommand{\headwidth}{\textwidth} 

% To adjust the frame length to the header and footer ones 
\autoslidemarginstrue 
\pagestyle{fancy} 
 

\newcommand{\heading}[1]{% 
  \begin{center} 
    \large\bf 
    #1 
  \end{center} 
  \vspace{.4 in}} 



\begin{document}

\centerslidesfalse 

\begin{slide}
\begin{footnotesize}
\begin{verbatim}
                          Existing Draft
      ClientHello
      ClientKeyShare            -------->
                                                      ServerHello
                                                   ServerKeyShare
                                           {EncryptedExtensions*}
                                                   {Certificate*}
                                            {CertificateRequest*}
                                             {CertificateVerify*}
                                <--------              {Finished}
      {Certificate*}
      {CertificateVerify*}
      {Finished}                -------->
      [Application Data]        <------->      [Application Data]

 {} Indicates messages protected using keys derived from the handshake
   master secret.

 [] Indicates messages protected using keys derived from the master
  secret.
\end{verbatim}
\end{footnotesize}
\end{slide}

\begin{slide}
\heading{Right now we have two modes}

\begin{itemize}
\item Signature-based (naive 1-RTT)
\item Static DH-based (cached server 1-RTT, 0-RTT)
\item Krawczyk/Wee propose just doing static DH
\end{itemize}
\end{slide}


\begin{slide}
\heading{Basic idea}

\begin{itemize}
\item Server has a semi-static DH key (just like 1-RTT)
\item Probably really has long-term signing key
  \begin{itemize}
  \item Used to sign the semi-static key
  \item .. offline versus online signing
  \end{itemize}

\item Key exchange computations the same for 0-RTT and 1-RTT (and similar with PSK)
\end{itemize}

\end{slide}


\begin{slide}
\begin{footnotesize}
\begin{verbatim}
                          DH-Based Draft

       ClientHello
       ClientKeyShare            -------->
                                                       ServerHello
                                                    ServerKeyShare
                                            {EncryptedExtensions*}
                                                    {Certificate*}
                                             {CertificateRequest*}
                                               {ServerParameters*}
                                 <--------              {Finished}
       {Certificate*}
       {CertificateVerify*}
       {Finished}                -------->
       [Application Data]        <------->      [Application Data]

 {} Indicates messages protected using keys derived from the handshake
   master secret.

 [] Indicates messages protected using keys derived from the master
  secret.
\end{verbatim}
\end{footnotesize}
\end{slide}



\begin{slide}
\heading{Server Parameters (prototype)}
\vspace{-.25in}
{\scriptsize
\begin{verbatim}
       struct {
         uint64 not_before;
         uint64 not_after;
         opaque key_identifier<0..2^8-1>;
         NamedGroup group;
         opaque server_key<1..2^16-1>;
       } UnsignedParameters;

       enum { online(0), (255)} ParametersType;
       
       struct {
           UnsignedParameters parameters;
           ParametersType params_type;
           digitally-signed struct {
             UnsignedParameters parameters;
             opaque session_hash[Hash.length];
           };
           opaque zero_rt_id<0..2^16-1>;
       } ServerParameters;

\end{verbatim}
}
\end{slide}

\begin{slide}
\heading{What's being signed by the server?}

\begin{itemize}
\item Online signature
  \begin{itemize}
  \item Rough consensus not to do offline now
  \end{itemize}
  
\item Signature over \verb^session_hash + UnsignedParameters^
\end{itemize}
\end{slide}


\begin{slide}
\vspace{-1.3in}
{\tiny
\begin{verbatim}
                                        0
                                        |                         
                                        |                         
                                      HKDF <- Static Secret<------+
                                     Extract
                                        |                         |
        Early Application               v                         |
          Traffic Keys    <-HKDFH--   Auth.                       |
                                     Master                       |
                                     Secret                       |
                                        |                         |
                0                       |                         | Via
                |                       |                         | Session
               HKDF <--  Ephemeral --> HKDF                       |
             Extract      Secret     Extract                      |
                |                       |                         |
                v                       |                         |
             Handshake                  |                         |
              Master                    |                         |
              Secret                    |                         |
                |                       |                         |
           HKDF-Expand                  |                         |
                |                       |                         |
                v                       v                         |
             Handshake                Master                      |
           Traffic Keys               Secret                      |
                                        |                         |
                                        |                         |
            Application    <--HKDF------+-----HKDF--->        Resumption        
           Traffic Keys      Expand     |    Expand             Master
                                        |                       Secret  
              Exporter        HKDF      |
           Master Secret   <--Expand----+
\end{verbatim}
}
\end{slide}

\begin{slide}
\heading{Performance Comparison (ignoring amortized computations and fixed base)}

\scriptsize{
\begin{tabular}{l l l}
Mode & Client & Server \\
\hline
Current TLS 1.3 (1-RTT) & Verify + 1 VB  & Sign + 1 VB  \\
OPTLS (online signing) & Verify + 2 VB  & Sign + 2 VB  \\
OPTLS (offline signing) & Verify + 2 VB  & 2 VB  \\
OPTLS (online signing, $g^s == g^y$) & Verify + 1 VB  & Sign + 1 VB  \\
OPTLS (HMQV) & Verify + 1.x VB  & Sign + 1.x VB  \\
0-RTT & 2 VB  & 2 VB  \\
\end{tabular}
}
\end{slide}


\begin{slide}
\heading{How do we provide keys for 0-RTT}

\begin{itemize}
\item Reuse/cache the keys from 1-RTT
  \begin{itemize}
  \item This means you can't use $g^s == g^y$
  \item Annoying tradeoff for the server
  \end{itemize}

\item Have the server supply a separate key
  \begin{itemize}
  \item Two keys in ServerParameters?
  \item Some other encoding
  \end{itemize}

\item Neither of these is that awesome
\end{itemize}
\end{slide}


\begin{slide}
\heading{HKDF}

\begin{itemize}
\item Some sense that we should convert to HKDF
  \begin{itemize}
  \item Clearer cryptographic properties
  \item Is there anything wrong with the TLS PRF?
  \end{itemize}

\item This is an orthogonal question
  \begin{itemize}
  \item But I'd like to make the change at the same time
  \end{itemize}
\end{itemize}
\end{slide}


\begin{slide}
\heading{Overview of 0-RTT Flow}

{\scriptsize
\begin{verbatim}
       Client                                               Server

       ClientHello
       ClientKeyShare
       {Certificate*}
       {CertificateVerify*}
       {Finished}              
       [Application Data]        -------->
                                                       ServerHello
                                                    ServerKeyShare
                                 <--------              {Finished}
       [Application Data]        <------->      [Application Data]
\end{verbatim}
}
\end{slide}


\begin{slide}
\heading{Key Agreement (well-understood)}

\begin{itemize}
\item Client caches server's long-term (EC)DH share
\item First flight data encrypted under client ephemeral/server static
\item Server supplies ephemeral in first flight
\item Rest of data encrypted under both ephemeral/static and ephemeral/ephemeral
\end{itemize}
\end{slide}


\begin{slide}
\heading{Anti-Replay}

\begin{itemize}
\item TLS anti-replay is based on each side providing random value
  \begin{itemize}
  \item Mixed into the keying material
  \end{itemize}

\item Not compatible with 0-RTT
  \begin{itemize}
  \item Client has anti-replay (since they speak first)
  \item Server's random isn't incorporated into client's first flight
  \end{itemize}
\end{itemize}
\end{slide}



\begin{slide}
\heading{Anti-Replay (borrowed from Snap Start)}

\begin{itemize}
\item Server needs to keep a list of client nonces
\item Indexed by a server-provided context token
\item Client provides a timestamp so server can maintain an anti-replay window
\end{itemize}
\end{slide}


\begin{slide}
\heading{This doesn't work}

\vspace{-.45in}
{\scriptsize
\begin{verbatim}
Client                        Attacker                          Server

ClientHello [+0-RTT] ------------>
"POST /buy-something" ----------->      

                              ClientHello [+0-RTT] --------------->
                              "POST /buy-something" -------------->      
                           
                                                   [Processes purchase]
                                 <---------- ServerHello [accept 0-RTT]
                                                  (+ rest of handshake)

                         [Force server reboot]

                              ClientHello [+0-RTT] --------------->
                              "POST /buy-something" -------------->      

   <--------------------------------------  ServerHello [reject 0-RTT]
                                                 (+ rest of handshake)

Finished --------------------------------------------------------->
"Post /buy-something" -------------------------------------------->
                                                  [Processes purchase]
\end{verbatim}
}
\end{slide}


\begin{slide}
\heading{Less Contrived}

\vspace{-.45in}
{\scriptsize
\begin{verbatim}
 Client             Attacker            Server1              Server 2

ClientHello [+0-RTT] -->
"POST /buy-something" ->               

                    ClientHello [+0-RTT] -->
                    "POST /buy-something" ->
                                       
                                        [Processes
                                         purchase]               



                    ClientHello [+0-RTT] ----------------------->
                    "POST /buy-something" ---------------------->


   <--------------------------------------  ServerHello [reject 0-RTT]
                                                 (+ rest of handshake)

Finished --------------------------------------------------------->
"Post /buy-something" -------------------------------------------->
                                                  [Processes purchase]
\end{verbatim}
}
\end{slide}


\end{document}

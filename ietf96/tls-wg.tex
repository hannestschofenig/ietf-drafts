\documentclass[helvetica]{seminar} 
\input{xy}
\xyoption{all}
\usepackage{graphicx} 
\usepackage{slidesec} 
\usepackage{url}
\usepackage[framemethod=TikZ]{mdframed}
\usepackage{color}
\usepackage[normalem]{ulem}  

\def\dash---{\unskip\kern.16667em---\penalty\exhyphenpenalty\hskip.16667em\ignorespaces}
\long\def\symbolfootnote[#1]#2{\begingroup%
\def\thefootnote{\fnsymbol{footnote}}\footnote[#1]{#2}\endgroup}

% to fix problems making landscape seminar pdfs
% Letter...
\pdfpagewidth=11truein
\pdfpageheight=8.5truein
\pdfhorigin=1truein     % default value(?), but doesn't work without
\pdfvorigin=1truein     % default value(?), but doesn't work without
% A4
%\pdfpagewidth=297truemm % your milage may vary....
%\pdfpageheight=210truemm
%\pdfhorigin=1truein     % default value(?), but doesn't work without
%\pdfvorigin=1truein     % default value(?), but doesn't work without


\renewcommand{\familydefault}{\sfdefault}  
 
\input{seminar.bug} 
\input{seminar.bg2} % See the Seminar bugs list 
 
\slideframe{none} 
 
 
\usepackage{fancyhdr} 
 
% Headers and footers personalization using the `fancyhdr' package 
\fancyhf{} % Clear all fields 
\renewcommand{\headrulewidth}{0mm} 
\renewcommand{\footrulewidth}{0.1mm} 
 
\fancyfoot[L]{\tiny IETF 96} 
\fancyfoot[C]{\tiny TLS}
\fancyfoot[R]{\tiny \theslide} 
 
 
% To center horizontally the headers and footers (see seminar.bug) 
\renewcommand{\headwidth}{\textwidth} 

% To adjust the frame length to the header and footer ones 
\autoslidemarginstrue 
\pagestyle{fancy} 
 

\newcommand{\heading}[1]{% 
  \begin{center} 
    \large\bf 
    #1 
  \end{center} 
  \vspace{.4 in}} 



\begin{document}

\begin{slide}
\begin{center}
\vspace{.5 in}
\LARGE{{\bf}TLS 1.3\\{\small \verb^draft-ietf-tls-tls13-14^}}\\
\vspace{.2in}
\large{
\begin{tabular}{c}
Eric Rescorla\\
Mozilla\\
\url{ekr@rtfm.com}
\end{tabular}
}
\end{center}
\end{slide}

\centerslidesfalse 

\begin{slide}
\heading{Major changes since draft-12}

\vspace{-8ex}
\begin{itemize}
\item Remove 0-RTT (EC)DHE and client auth *
\item Complete 0-RTT PSK mode *
\item Restructure key schedule *
\item Add session context *
\item Fully define HelloRetryRequest *
\item NewSession ticket use flags
\item Allow server to send SupportedGroups
\item Move CertificateStatus to an extension
\item Add ticket age for anti-replay
\item Allow resumption after fatal alerts
\item Remove non-closure warning alerts
\item Add Security Analysis section
\end{itemize}

\end{slide}


\begin{slide}
\heading{0-RTT is now PSK-only}

\vspace{-3ex}
\begin{scriptsize}
\begin{verbatim}
            ClientHello
              + early_data
              + pre_shared_key
              + key_share*
            (Finished)
            (Application Data*)
            (end_of_early_data)       -------->
                                                            ServerHello
                                                           + early_data
                                                       + pre_shared_key
                                                           + key_share*
                                                  {EncryptedExtensions}
                                                  {CertificateRequest*}
                                                             {Finished}
                                      <--------     [Application Data*]
            {Certificate*}
            {CertificateVerify*}
            {Finished}                -------->
            [Application Data]        <------->      [Application Data]
\end{verbatim}
\end{scriptsize}
\end{slide}

\begin{slide}
\vspace{-8ex}
{\tiny
\begin{verbatim}
                 0
                 |
                 v
   PSK ->  HKDF-Extract
                 |              
                 v
           Early Secret  --> Derive-Secret(., "early traffic secret", ClientHello)
                 |                         = early_traffic_secret
                 v
(EC)DHE -> HKDF-Extract
                 |
                 v
              Handshake
               Secret -----> Derive-Secret(., "handshake traffic secret", ClientHello + ServerHello)
                 |                         = handshake_traffic_secret
                 v
      0 -> HKDF-Extract
                 |
                 v
            Master Secret
                 |
                 +---------> Derive-Secret(., "application traffic secret", ClientHello...Server Finished)
                 |                         = traffic_secret_0
                 |
                 +---------> Derive-Secret(., "exporter master secret", ClientHello...Client Finished)
                 |                         = exporter_secret
                 |
                 +---------> Derive-Secret(., "resumption master secret", ClientHello...Client Finished)
                                           = resumption_secret


\end{verbatim}
}
\end{slide}

\begin{slide}
\heading{Session Context}

\begin{itemize}
\item Multiple requests to include more context when resuming (Krawczyk, Bhargavan)
\end{itemize}

{\footnotesize
\begin{verbatim}
     resumption_psk = HKDF-Expand-Label(resumption_secret,
                                        "resumption psk", "", L)

     resumption_context = HKDF-Expand-Label(resumption_secret,
                                            "resumption context", "", L)
\end{verbatim}
}

\begin{itemize}
\item Merged into handshake hashes whenever used
\end{itemize}

{\footnotesize
\begin{verbatim}
     Hash(Messages) + Hash(resumption_context)
\end{verbatim}
}
\end{slide}


\begin{slide}
\heading{Cookies for HelloRetryRequest}

\begin{itemize}
\item Derived from DTLS (and originally Photuris)
\item Server can provide a cookie with HRR 
\item Client echoes it with new ClientHello
\item Usable for stateless reject by pickling the handshake state in the cookie
\end{itemize}
\end{slide}


\begin{slide}
\heading{Post-Handshake Key Separation}

\begin{itemize}
\item General consensus on list to leave as-is
\item Analysis from Hugo Krawczyk indicates this is OK
\item IMPORTANT: We still have key separation for ordinary-handshake and app data
\end{itemize}
\end{slide}


\begin{slide}
\heading{Cipher Suite Negotiation: Problem Statement}

\begin{itemize}
\item The cipher suite negotiation has gotten clunky and non-orthogonal
\item Already was bad in 1.2
  \begin{itemize}
  \item Cipher suite, signature algorithms, named groups
  \end{itemize}
\item Worse in 1.3 
  \begin{itemize}
  \item PSK, key shares
  \end{itemize}
\item Can we radically simplify?
\end{itemize}
\end{slide}

\begin{slide}
\heading{Cipher Suite Negotiation: Overview}

\begin{itemize}
\item Break up into the following axes
  \begin{itemize}
  \item AEAD-PRF
  \item Signature algorithms
  \item Key shares/named groups
  \item PSK
  \end{itemize}

\item Negotiate each separately
  \begin{itemize}
  \item Straightforward for public key
  \item PSK makes things a bit complicated
  \end{itemize}
\end{itemize}
\end{slide}

\begin{slide}
\heading{Public key algorithm negotiation}

\begin{itemize}
\item Cipher suite just indicates AEAD and PRF
  \begin{itemize}
  \item Probably define new cipher suites
  \item Added bonus of letting us prune!
  \end{itemize}

\item Signature algorithms determines server cert/key and signature scheme
\item Key shares and supported groups determine the key exchange
  \begin{itemize}
  \item Model everything as (EC)DHE
  \item Server's key share indicates which group it picked
  \end{itemize}
\end{itemize}
\end{slide}

\begin{slide}
\heading{What about PSK?}

\vspace{-5ex}
\begin{itemize}
\item PSK can be combined with (EC)DHE and signatures (new) (?)
\end{itemize}

\vspace{-1ex}
\begin{scriptsize}
\begin{verbatim}
    enum { psk_ke(0), psk_dhe_ke(1), (255) } PskKeModes;
    enum { psk_auth(0), psk_sign_auth(1), (255) } PskAuthModes;

    struct {
       PskAuthMode auth_modes<1..255>;
       PskKeMode ke_modes<1..255>;
       opaque identity<0..2^16-1>; 
    } PskIdentity;

    struct {
         select (Role) {
             case client:
                 PskIdentity identities<2..2^16-1>;
              case server:
                 PskAuthMode auth_mode;
                 PskKeMode ke_mode;
                 uint16 selected_identity;
         }
     } PreSharedKeyExtension;
\end{verbatim}
\end{scriptsize}
\end{slide}


\begin{slide}
\heading{Should we change negotiation?}

\begin{itemize}
\item Cons
  \begin{itemize}
  \item Big change at the last minute
  \item Makes APIs more complicated because the cipher suite doesn't tell you everything
  \item Doesn't let you express non-orthogonal options
  \end{itemize}

\item Pros
  \begin{itemize}
  \item Much easier to implement (based on initial prototypes)
  \item Removes odd pairing of (EC)DHE and PSK cipher suites
  \item More expressive
  \end{itemize}

\item Proposal: provisionally adopt pending a PR
\end{itemize}

\end{slide}


\begin{slide}
\heading{Version Negotiation}

\includegraphics[width=4in]{932382}

\end{slide}

\begin{slide}
\heading{Alternate Proposal}

\begin{itemize}
\item Keep \verb^ClientHello^ version number at {3, 3} (TLS 1.2)

\item Introduce a new tls\_version extension
  \begin{itemize}
  \item Semantic is: a list of all supported versions
  \item Example: \verb^[ [3, 2], [3, 3], [3, 4], [53, 100] ]^
  \end{itemize}

\item ServerHello contains the negotiated version

\item All future versions negotiated this way
\begin{itemize}
\item Can fuzz for futureproofing
\end{itemize}

\item Discuss
\end{itemize}

\end{slide}


\begin{slide}
\heading{PSK and Client Auth}

\vspace{-4ex}
\begin{itemize}
\item Draft implies support for client authentication even with PSK mode
  \begin{itemize}
  \item Server just sends CertificateRequest
  \item Semantics of this are odd.
  \item 0-RTT is even worse
  \end{itemize}

\item Main proposal
  \begin{itemize}
  \item CertificateRequest not allowed when using PSK
  \item Use post-handshake client auth if you want this
  \end{itemize}

\item Fallback proposal
  \begin{itemize}
  \item PSK client auth needs an identity that is ``morally the same''
  \item Then clients can refuse to refresh
  \end{itemize}

\item Proposed resolution: ban client auth PSK
\end{itemize}

\end{slide}


\begin{slide}
\heading{Resumption Contexts and 0-RTT Finished}

\begin{itemize}
\item From the 0-RTT Finished:
\begin{itemize}
\item Proof of at least partial liveness of the PSK [via ticket age]
\item An integrity check for the information in the ClientHello
\end{itemize}
\item From the resumption context:
\begin{itemize}
\item Tie the context from the PSK-establishing connection to
  future handshakes.
\end{itemize}

\item Issues
  \begin{itemize}
  \item ``0'' resumption\_context for out-of-band PSK is problematic
  \item This seems duplicative
  \item Reading the 0-RTT Finished is kind off a pain
  \item Always adding the PSK context to the hash is clunky
  \end{itemize}
\end{itemize}

\end{slide}


\begin{slide}
\heading{Potential Options}

\begin{itemize}
\item Remove 0-RTT Finished but use it as resumption\_ctx
  \begin{itemize}
  \item \verb^resumption_ctx = HMAC(., ClientHello)^
  \end{itemize}
\item \sout{Always require 0-RTT Finished even w/o 0-RTT (and include in the log)}
\item Always include a special Finished extension when using PSK
  \begin{itemize}
  \item And discard \verb^resumption_ctx^
  \item This can be a bit tricky to implement
  \end{itemize}
\item Do nothing
\item[]
\item Proposal: ???
\end{itemize}
\end{slide}

\begin{slide}
\heading{Crypto for Embedded 0-RTT Finished (thanks to Antoine)}

\begin{verbatim}
Early Secret = HKDF-Extract(0, PSK)
early_finished_secret =
     Derive-Secret(Early Secret, "...", ClientHello-prefix)
ClientHello = ClientHello + HMAC(efs, ClientHello-prefix)
early_traffic_secret =
     Derive-Secret(Early Secret, "...", ClientHello)
\end{verbatim}
\end{slide}


\begin{slide}
\heading{Multiple Concurrent Tickets (PR \#8)}

\begin{itemize}
\item Currently we implicitly support multiple tickets
  \begin{itemize}
  \item Useful for de-linkage privacy, etc.
  \end{itemize}

\item Ticket encoding gives no guidance about how to use them
  \begin{itemize}
  \item Is ticket $N$ usable after I see ticket $N+1$? Try it and see!
  \end{itemize}

\item Proposal: Add a field (generation?) to indicate whether a ticket supersedes others
\end{itemize}

\end{slide}


\begin{slide}
\heading{Interop Status}

\includegraphics[width=4in]{interop-matrix}

\end{slide}


\begin{slide}
\heading{Timeline: Option \#1 (No big changes)}

\begin{tabular}{l l}
Aug 8th & draft-15: Wire format frozen (``Cryptographer's version'') \\
Aug 22nd & Implementations of draft-15 \\
Aug 29th & draft-16: Revised based on feedback \\
Aug 29th & WGLC \\
Sep 30th & WGLC Ends \\
\end{tabular}

\end{slide}


\begin{slide}
\heading{Timeline: Option \#2 (Change Negotiation or 0-RTT Finished)}

\begin{tabular}{l l}
Aug 8th & draft-15: Changes agreed at IETF 96 \\
Aug 22nd & Implementations of draft-15 \\
Aug 29th & draft-16: Revised; Wire format frozen (``cryptographer's version'')\\
Sep 12th & Implementations of draft-16 \\
Sep 19th & draft-17: Revised based on feedback \\
Sep 19th & WGLC \\
Oct 17th & WGLC Ends \\
\end{tabular}

\end{slide}


\end{document} 

                
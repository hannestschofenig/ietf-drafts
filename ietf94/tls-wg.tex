\documentclass[helvetica]{seminar} 
\input{xy}
\xyoption{all}
\usepackage{graphicx} 
\usepackage{slidesec} 
\usepackage{url}
\usepackage[framemethod=TikZ]{mdframed}
\usepackage{color}

\long\def\symbolfootnote[#1]#2{\begingroup%
\def\thefootnote{\fnsymbol{footnote}}\footnote[#1]{#2}\endgroup}

% to fix problems making landscape seminar pdfs
% Letter...
\pdfpagewidth=11truein
\pdfpageheight=8.5truein
\pdfhorigin=1truein     % default value(?), but doesn't work without
\pdfvorigin=1truein     % default value(?), but doesn't work without
% A4
%\pdfpagewidth=297truemm % your milage may vary....
%\pdfpageheight=210truemm
%\pdfhorigin=1truein     % default value(?), but doesn't work without
%\pdfvorigin=1truein     % default value(?), but doesn't work without


\renewcommand{\familydefault}{\sfdefault}  
 
\input{seminar.bug} 
\input{seminar.bg2} % See the Seminar bugs list 
 
\slideframe{none} 
 
 
\usepackage{fancyhdr} 
 
% Headers and footers personalization using the `fancyhdr' package 
\fancyhf{} % Clear all fields 
\renewcommand{\headrulewidth}{0mm} 
\renewcommand{\footrulewidth}{0.1mm} 
 
\fancyfoot[L]{\tiny IETF 94} 
\fancyfoot[C]{\tiny TLS}
\fancyfoot[R]{\tiny \theslide} 
 
 
% To center horizontally the headers and footers (see seminar.bug) 
\renewcommand{\headwidth}{\textwidth} 

% To adjust the frame length to the header and footer ones 
\autoslidemarginstrue 
\pagestyle{fancy} 
 

\newcommand{\heading}[1]{% 
  \begin{center} 
    \large\bf 
    #1 
  \end{center} 
  \vspace{.4 in}} 



\begin{document}

\begin{slide}
\begin{center}
\vspace{.5 in}
\LARGE{{\bf}TLS 1.3 Status\\draft-10}\\
\vspace{.2in}
\large{
\begin{tabular}{c}
Eric Rescorla\\
Mozilla\\
\url{ekr@rtfm.com}
\end{tabular}
}
\end{center}

\end{slide}

\centerslidesfalse 


\begin{slide}
\heading{Overview}

\begin{itemize}
\item Changes since IETF 93 (Prague)
\item Client authentication (PR\#316)
\item 0-RTT framing (\#311, \#295)
\item HelloRetryRequest (Issues \#104, \#185)
\item Re-key (\#4, \#125)
\item Exporters (\#282)
\end{itemize}

\end{slide}

\begin{slide}
\heading{Changes Since IETF 93 (II)}

\begin{itemize}
\item Always require digital signatures from the server with public-key cipher suites
  \begin{itemize}
  \item ...even with 0-RTT
  \end{itemize}

\item Relaxed certificate selection rules *
\item Deprecated a lot of algorithms *
\item Encrypted content type *
\item Built-in record padding *
\item More context for key derivation *
\item Improved CertificateRequest syntax *
\end{itemize}
\end{slide}


\begin{slide}
\heading{Changes Since IETF 93 (II)}

\begin{itemize}
\item Update key schedule
\item Added MTI algorithms
\item Reduced maximum record expansion
\item Extensionsify ServerKeyShare
\item AEAD now has no AAD
\item Assorted editorial stuff
\end{itemize}
\end{slide}


\begin{slide}
\heading{Relaxed Certificate Selection Rules}

\begin{itemize}
\item TLS 1.2 requires that certificates appear in order
  \begin{itemize}
  \item Many servers don't do this
    \begin{itemize}
    \item Not always possible
    \end{itemize}
  \item Many clients try to construct the path anyway
  \item Updated draft to encourage but not require this
  \end{itemize}

\item TLS 1.2 required that server certificates conform to SignatureAlgorithms
  \begin{itemize}
  \item But what if the only cert you have doesn't match?
  \item Draft now allows you to send it in that case 
    \begin{itemize}
    \item ...but only if you have to
    \end{itemize}
  \end{itemize}
\end{itemize}
\end{slide}


\begin{slide}
\heading{Deprecated Algorithms} 

\begin{itemize}
\item Forbid MD5 (and SHA-224)
\item Forbid SHA-1 in CertificateVerify
\item Removed DSA
\item Switched to PSS (more on this later)
\item Removed a lot of old EC groups
\end{itemize}

\end{slide}

\begin{slide}
\heading{Encrypted Content Type and Padding}

\begin{footnotesize}
\begin{verbatim}
   struct {
       ContentType opaque_type = application_data(23); /* see fragment.type */
       ProtocolVersion record_version = { 3, 1 };    /* TLS v1.x */
       uint16 length;
       aead-ciphered struct {
          opaque content[TLSPlaintext.length];
          ContentType type;
          uint8 zeros[length_of_padding];
       } fragment;
   } TLSCiphertext;
\end{verbatim}
\end{footnotesize}

\begin{itemize}
\item This allows padding
\item But doesn't require it
\item Receiver behaves the same either way
\end{itemize}
\end{slide}


\begin{slide}
\heading{Improved CertificateRequest Syntax (Popov)}

\vspace{-3ex}
\begin{footnotesize}
\begin{verbatim}
      struct {
          opaque certificate_extension_oid<1..2^8-1>;
          opaque certificate_extension_values<0..2^16-1>;
      } CertificateExtension;

      struct {
          SignatureAndHashAlgorithm
            supported_signature_algorithms<2..2^16-2>;
          DistinguishedName certificate_authorities<0..2^16-1>;
          CertificateExtension certificate_extensions<0..2^16-1>;
      } CertificateRequest;
\end{verbatim}
\end{footnotesize}

\begin{itemize}
\item Extensions correspond to X.509v3 extensions in the EE certificate
\item Each extension has its own matching rule
  \begin{itemize}
  \item KeyUsage and EKU defined in this document
  \end{itemize}
\item Client can ignore any unrecognized extensions
\end{itemize}
\end{slide}


\begin{slide}
\heading{Client Authentication (PR\#316)}

\begin{itemize}
\item TLS 1.3 removed renegotiation
\item But there's still a need for servers to request certificates post-handshake
  \begin{itemize}
  \item Especially in HTTP
  \end{itemize}

\item WG had consensus in Seattle to do something about this
\item Formed ad hoc design team
  \begin{itemize}
  \item AGL, DKG, EKR, Beurdouche, Bhargavan, Krawczyk, Langley, MT, Wee
  \end{itemize}
\end{itemize}
\end{slide}

\begin{slide}
\heading{Current Structure}

\vspace{-3ex}
\begin{footnotesize}
\begin{verbatim}
           ClientHello
             + ClientKeyShare     -------->
                                                        ServerHello
                                                    ServerKeyShare*
                                              {EncryptedExtensions}
                                             {ServerConfiguration*}
                                                     {Certificate*} <-\
                                              {CertificateRequest*}    > Sign.
                                               {CertificateVerify*} <-/
                                  <--------              {Finished} <-   MAC
Sign.  /-> {Certificate*}
       \-> {CertificateVerify*}
MAC     -> {Finished}             -------->
           [Application Data]     <------->      [Application Data]
\end{verbatim}
\end{footnotesize}

\begin{itemize}
\item This is effectively SIGMA-I
\item So what if we formalize it
\end{itemize}
\end{slide}

\begin{slide}
\heading{TLS Authentication Block}

\begin{itemize}
\item Consists of: Certificate, CertificateVerify, Finished
  \begin{itemize}
  \item Use this every time we want to authenticate
  \item Sometimes Cert/CertVerify are omitted
  \end{itemize}

\item Inputs are:
  \begin{itemize}
  \item A Session Context (usually the handshake transcript)
  \item A base key to compute the finished keys from
    \begin{itemize}
    \item Client and server use separate keys
    \end{itemize}
  \end{itemize}

\item CertificateVerify = Sign(SC + Certificate)
\item Finished = MAC(SC + Certificate + CertificateVerify)
  \begin{itemize}
  \item Note: this is like continuing the hashes
  \end{itemize}
\end{itemize}
\end{slide}


\begin{slide}
\heading{Authentication Inputs}

\vspace{-3ex}
\begin{footnotesize}
\begin{verbatim}
Mode             Handshake Context                        Base Key
----             -----------------                        --------
0-RTT            ClientHello + ServerConfiguration        xSS
                             + Server Certificate
                             + CertificateRequest
                            
1-RTT (Server)   ClientHello ... ServerConfiguration      master_secret
                 
1-RTT (Client)   ClientHello ... ServerFinished           master_secret

Post-Handshake   ClientHello ... ClientFinished +         master_secret
                 CertificateRequest
\end{verbatim}
\end{footnotesize}
\end{slide}

\begin{slide}
\heading{Post-Handshake Client Auth}

\begin{itemize}
\item Server can send CertificateRequest at any time
\item Client responds with authentication block
  \begin{itemize}
  \item Possibly with empty cert
  \end{itemize}

\item Note: need to add correlator between CertificateRequest and CertificateVerify
\item Not in this PR yet
\end{itemize}

\end{slide}

\begin{slide}
\heading{Key Schedule Changes}

\begin{footnotesize}
\begin{verbatim}
  3. mSS = HKDF-Expand-Label(xSS, "expanded static secret",
                             handshake_hash, L)

  4. mES = HKDF-Expand-Label(xES, "expanded ephemeral secret",
                             handshake_hash, L)

  Where handshake_hash includes all messages up through the
  server CertificateVerify message.

  5. master_secret = HKDF-Extract(mSS, mES)

  client_finished_key =
      HKDF-Expand-Label(BaseKey, "client_finished", "", L)
  
  server_finished_key =
      HKDF-Expand-Label(BaseKey, "server_finished", "", L)
\end{verbatim}
\end{footnotesize}
\end{slide}

\begin{slide}
\begin{scriptsize}
\begin{verbatim}
            ClientHello
              + ClientKeyShare
         ^    + EarlyDataIndication
  O-RTT  |  (Certificate*)
  mode   |  (CertificateVerify*
         v  (Finished)  // Note: new message.
            (Application Data*)       -------->
                                                            ServerHello
                                                        ServerKeyShare*
                                                  {EncryptedExtensions}
                                                  {CertificateRequest*}
                                                 {ServerConfiguration*} 
                                                         {Certificate*}  ^
                                                   {CertificateVerify*}  | Server Auth.
                                      <--------              {Finished}  v
 1-RTT   ^  {Certificate*}
 Client  |  {CertificateVerify*}
 Auth    |  {Finished}                -------->
         v  [Application Data]        <------->      [Application Data]

                         
                                      <--------    [CertificateRequest]   ^
            [Certificate]                                                 | Post-HS
            [CertificateVerify]                                           | Auth.
            [Finished]                -------->                           v
\end{verbatim}
\end{scriptsize}
\end{slide}

\begin{slide}
\heading{Other Notes}

\begin{itemize}
\item Added Finished to 0-RTT data
  \begin{itemize}
  \item It's part of authentication block
  \item Adds consistency and a natural separator
  \end{itemize}

\item 0-RTT data isn't hashed into transcript for 1-RTT
  \begin{itemize}
  \item Conceptually cleaner to separate these
  \item Not necessary for negotiation
  \end{itemize}

\item Possible to client authenticate \emph{both} in 0-RTT and 1-RTT
  \begin{itemize}
  \item Conceptually simpler
  \item Server can keep requesting anyway
  \end{itemize}
\end{itemize}
\end{slide}

\begin{slide}
  \heading{Framing for 0-RTT(\#311, \#295)}

  \begin{itemize}
  \item 0-RTT content types are funny
    \begin{itemize}
    \item Handshake uses ``early\_data''
    \item Application uses ``application\_data''
    \end{itemize}
  \item Idea was to separate by content type
    \begin{itemize}
    \item Even without keys
    \end{itemize}

  \item This doesn't work with encrypted content types
  \item Proposed resolution
    \begin{itemize}
    \item 0-RTT content uses the expected content types
    \item Terminate 0-RTT application data with close\_notify
    \item Recovering from a failed 0-RTT requires trial decryption
    \end{itemize}
  \end{itemize}
\end{slide}

\begin{slide}
  \heading{HelloRetryRequest and Handshake Hash (\#104, \#185}

  \begin{itemize}
  \item Document is agnostic about handshake hash when HRR is used
  \item Option 1: Continue hash
    \begin{itemize}
    \item Much easier to analyze for handshake correctness
    \item But we want the HRR to be stateless
      \begin{itemize}
      \item Combine HRR with DTLS cookie exchange
      \end{itemize}
    \end{itemize}

  \item Option 2: Reset hash
    \begin{itemize}
    \item Easy to make stateless
    \item Much harder to analyze
    \end{itemize}

  \item It turns out we can have both good properties
  \end{itemize}
\end{slide}

\begin{slide}
\heading{Stateless HelloRetryRequest}

\vspace{-3ex}
\begin{itemize}
\item Import cookie exchange from DTLS
  \begin{itemize}
  \item Server sends a cookie with HRR
  \item Client echoes back cookie with new Hello
  \end{itemize}

\item Retain existing rules for repeat ClientHello construction
  \begin{itemize}
  \item Append new ClientKeyShare (if needed)
  \item Add cookie
  \item No other changes
  \end{itemize}

\item Server can recover the handshake hash state
  \begin{itemize}
  \item Option 1: offload state into cookie (integrity protected)
  \item Option 2: reconstruct the ClientHello from the rules above
  \item Option 3: Or just keep state (makes sense in TLS)
  \end{itemize}
\item This is all invisible to the client
\end{itemize}
\end{slide}

\begin{slide}
\heading{Other cookie construction issues}

\begin{itemize}
\item Cookie should indicate why HRR was sent
  \begin{itemize}
  \item Needed for Option\#2.
  \item Can still be opaque
  \end{itemize}
\item Want to allow use of cookie as ``address token''
  \begin{itemize}
  \item Client can send it repeatedly
  \item Do we need structure in the cookie to indicate that?
  \end{itemize}
\end{itemize}
\end{slide}

\end{document}  
                
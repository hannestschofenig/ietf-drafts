\documentclass[helvetica]{seminar} 
\input{xy}
\xyoption{all}
\usepackage{graphicx} 
\usepackage{slidesec} 
\usepackage{url}
\usepackage[framemethod=TikZ]{mdframed}

\long\def\symbolfootnote[#1]#2{\begingroup%
\def\thefootnote{\fnsymbol{footnote}}\footnote[#1]{#2}\endgroup}

% to fix problems making landscape seminar pdfs
% Letter...
\pdfpagewidth=11truein
\pdfpageheight=8.5truein
\pdfhorigin=1truein     % default value(?), but doesn't work without
\pdfvorigin=1truein     % default value(?), but doesn't work without
% A4
%\pdfpagewidth=297truemm % your milage may vary....
%\pdfpageheight=210truemm
%\pdfhorigin=1truein     % default value(?), but doesn't work without
%\pdfvorigin=1truein     % default value(?), but doesn't work without



\renewcommand{\familydefault}{\sfdefault}  
 
\input{seminar.bug} 
\input{seminar.bg2} % See the Seminar bugs list 
 
\slideframe{none} 
 
 
\usepackage{fancyhdr} 
 
% Headers and footers personalization using the `fancyhdr' package 
\fancyhf{} % Clear all fields 
\renewcommand{\headrulewidth}{0mm} 
\renewcommand{\footrulewidth}{0.1mm} 
 
\fancyfoot[L]{\tiny IETF 89} 
\fancyfoot[C]{\tiny HPACK Security Overview}
\fancyfoot[R]{\tiny \theslide} 
 
 
% To center horizontally the headers and footers (see seminar.bug) 
\renewcommand{\headwidth}{\textwidth} 

% To adjust the frame length to the header and footer ones 
\autoslidemarginstrue 
\pagestyle{fancy} 
 

\newcommand{\heading}[1]{% 
  \begin{center} 
    \large\bf 
    #1 
  \end{center} 
  \vspace{.4 in}} 


\begin{document}
\begin{slide}
\begin{center}
\vspace{.5 in}
\LARGE{{\bf}HPACK Security Observations}\\
\vspace{.2in}
\vspace{3em}
\large{
\begin{tabular}{c}
Eric Rescorla\\
Mozilla\\
\url{ekr@rtfm.com}
\end{tabular}
}
\end{center}

\end{slide}

\centerslidesfalse 


\begin{slide}
\heading{Threat Model Overview (Abstract)}

\begin{itemize}
\item Compression function acts as an oracle
\item Attacker can \emph{query} the oracle to see if a given header/value is present
\item Gets one bit in response: \emph{was this value seen before?}
\item[]
\item What can the attacker do?
  \begin{itemize}
  \item Search the space of a given header
  \item Only works with low entropy secrets
  \end{itemize}
\end{itemize}
\end{slide}


\begin{slide}
\heading{Is there any way to exploit this in HTTPS?}

\begin{itemize}
\item Cookies are pretty high entropy
\item But what about passwords?
  \begin{itemize}
  \item Often very low entropy
  \item Appear in headers with basic auth (how commonly is this used?)
  \item Absent CORS JS probably can't modify this header
  \end{itemize}
\item What about other headers?
\item And other technologies? (Flash, etc.)
\end{itemize}
\end{slide}


\begin{slide}
\heading{Why is this stronger than just querying the server?}

\vspace{-.55in}
$$
\xymatrix@C=1.2in@R=.2in{
  Client & Attacker & Server \\
  & \ar[l]_{\txt{Attack Code}} \\
  \ar@{<->}[rr]^{\txt{TLS Handshake}} & & \\
  \ar@{->}[r]^{\txt{Requests with guessed secret }} & X \\
}
$$

\begin{itemize}
\item Most servers have mechanisms to prevent fast guessing attacks
  \begin{itemize}
  \item Rate limiting, limited try, etc.
  \end{itemize}

\item The attacker allows the client and server to set up an HTTP2/TLS connection
\item The attacker injects queries with their guesses
  \begin{itemize}
  \item But blocks the client's requests to the server
  \item So the server never sees the guesses
  \end{itemize}

\end{itemize}

\end{slide}


\begin{slide}
\heading{Some words from Adam Barth}

\footnotesize{
\begin{quote}
``The situation gets worse if we consider non-standard web technology,
such as Flash.  For example, Flash's URLRequest API lets the attacker
set a wide variety of headers because it uses a header blacklist
rather than a whitelist [2].  Worse, Flash permits the attacker to
issue such requests across origins via the navigateToURL API.  It just
so happens that the Authorization header is on Flash's header
blacklist, but we need to consider the possibility that web sites will
store sensitive information in headers that aren't on Flash's
blacklist.

One reaction I can imagine to this issue is to blame Flash and decry
its use of a blacklist rather than a whitelist for security, but that
misses the larger point that HPACK weakens security because it
requires all downstream technologies to maintain more invariants in
order to avoid leaking sensitive information out of an otherwise
secure channel.''
\end{quote}
}

\end{slide}

\end{document}

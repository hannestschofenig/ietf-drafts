\documentclass[helvetica]{seminar} 
\input{xy}
\xyoption{all}
\usepackage{graphicx} 
\usepackage{slidesec} 
\usepackage{url}
\usepackage[framemethod=TikZ]{mdframed}

\long\def\symbolfootnote[#1]#2{\begingroup%
\def\thefootnote{\fnsymbol{footnote}}\footnote[#1]{#2}\endgroup}

% to fix problems making landscape seminar pdfs
% Letter...
\pdfpagewidth=11truein
\pdfpageheight=8.5truein
\pdfhorigin=1truein     % default value(?), but doesn't work without
\pdfvorigin=1truein     % default value(?), but doesn't work without
% A4
%\pdfpagewidth=297truemm % your milage may vary....
%\pdfpageheight=210truemm
%\pdfhorigin=1truein     % default value(?), but doesn't work without
%\pdfvorigin=1truein     % default value(?), but doesn't work without



\renewcommand{\familydefault}{\sfdefault}  
 
\input{seminar.bug} 
\input{seminar.bg2} % See the Seminar bugs list 
 
\slideframe{none} 
 
 
\usepackage{fancyhdr} 
 
% Headers and footers personalization using the `fancyhdr' package 
\fancyhf{} % Clear all fields 
\renewcommand{\headrulewidth}{0mm} 
\renewcommand{\footrulewidth}{0.1mm} 
 
\fancyfoot[L]{\tiny IETF 89} 
\fancyfoot[C]{\tiny DTLS and IPsec for DNSE}
\fancyfoot[R]{\tiny \theslide} 
 
 
% To center horizontally the headers and footers (see seminar.bug) 
\renewcommand{\headwidth}{\textwidth} 

% To adjust the frame length to the header and footer ones 
\autoslidemarginstrue 
\pagestyle{fancy} 
 

\newcommand{\heading}[1]{% 
  \begin{center} 
    \large\bf 
    #1 
  \end{center} 
  \vspace{.4 in}} 


\begin{document}
\begin{slide}
\begin{center}
\vspace{.5 in}
\LARGE{{\bf}DTLS and IPsec for DNSE}\\
\vspace{.2in}
\vspace{3em}
\large{
\begin{tabular}{c}
Eric Rescorla\\
Mozilla\\
\url{ekr@rtfm.com}
\end{tabular}
}
\end{center}

\end{slide}

\centerslidesfalse 


\begin{slide}
\heading{Basic assumptions}

\begin{itemize}
\item We have some other mechanism for identifying the expected server key
  \begin{itemize}
  \item Manual configuration for configured resolvers
  \item DNS records for authoritative servers
  \end{itemize}

\item Client is anonymous
\end{itemize}

\end{slide}


\begin{slide}
\heading{General Pattern (current)}

\vspace{-.55in}
$$
\xymatrix@C=1.2in@R=.2in{
  Client & & Server \\
  \ar@{<->}[rr]^{\txt{Handshake/Setup}}& & \\
  \ar[rr]^{\txt{DNS Request}} & &\\
  & & \ar[ll]_{\txt{DNS Response}}
 & ...
}
$$

\begin{itemize}
\item What is this handshake thing?
\begin{itemize}
\item DTLS and IPsec both need a cryptographic handshake to set up the parameters and exchange keys
\end{itemize}
\item Once keys are exchanged, you just send UDP
\begin{itemize}
\item DTLS: DNS Message/DTLS record/UDP/IP
\item IPsec: DNS Message/UDP/IPsec packet
\end{itemize}
\end{itemize}
\end{slide}


\begin{slide}
\heading{Handshake Issues: NAT/Firewall Penetration}

\begin{itemize}
\item The handshake needs to get through middleboxes
\item Both handshakes are carried over UDP
  \begin{itemize}
  \item DTLS on the same port
  \item IPsec on a separate (defined) port
  \end{itemize}

\item We have to worry about these UDP packets being blocked
  \begin{itemize}
    \item Lots of people block non-53 UDP
      \begin{itemize}
      \item Though maybe pass IKE
      \end{itemize}
    \item But there are also protocol enforcement devices that enforce things look like DNS
  \end{itemize}

\item Also need to worry about middleboxes blocking ESP
\end{itemize}
\end{slide}

\begin{slide}
\heading{Handshake Latency}

\begin{itemize}
\item The handshake isn't free (1-2 RTT)
\item Not a big problem for connections to your recursive resolver
  \begin{itemize}
  \item Just do a setup at bootup and then send requests over it
  \item Though requires storing state at the server
  \end{itemize}

\item Less good when you are the recursive resolver
  \begin{itemize}
  \item Need to handshake for each new resolver you see
  \item Current option: bear the cost of the handshake each time
  \item Future options: TLS WG is currently designing 0-RTT modes for TLS 1.3 which might help
  \end{itemize}
\end{itemize}
\end{slide}


\begin{slide}
\heading{Final Thoughts}

\begin{itemize}
\item IPsec probably isn't going to work here
\item DTLS might work
  \begin{itemize}
  \item But potential firewall and performance issues need research
  \end{itemize}

\item Happy to help if people want to talk more
\end{itemize}

\end{slide}

\end{document}

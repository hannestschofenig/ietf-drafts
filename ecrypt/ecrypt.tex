\documentclass[semrot,fancybox]{seminar}

\usepackage{ifpdf}
\usepackage{graphicx}
\input xy
\xyoption{all}
\ifpdf
%%%%%%%%
% to fix problems making landscape seminar pdfs
% Letter...
\pdfpagewidth=11truein
\pdfpageheight=8.5truein
% A4
%\pdfpagewidth=297truemm % your milage may vary....
%\pdfpageheight=210truemm
\pdfhorigin=1truein     % default value(?), but doesn't work without
\pdfvorigin=1truein     % default value(?), but doesn't work without
\fi

\usepackage{graphicx}
\usepackage{color} 
\usepackage{slidesec}
\renewcommand{\familydefault}{\sfdefault}  

\input{seminar.bug}
\input{seminar.bg2} % See the Seminar bugs list

\slideframe{none}
\centerslidesfalse


\usepackage{fancyhdr}

% Headers and footers personalization using the `fancyhdr' package
\fancyhf{} % Clear all fields
\renewcommand{\headrulewidth}{0mm}
\renewcommand{\footrulewidth}{0.1mm}

\fancyfoot[L]{\tiny Eric Rescorla}
\fancyfoot[C]{\tiny ECRYPT Hash Workshop 2007}
\fancyfoot[R]{\tiny \theslide}


\newcommand{\heading}[1]{%
  \begin{center}
    \large\bf
    #1
  \end{center}
  \vspace{.4 in}}


\begin{document} 

\begin{slide}

\begin{center}
\LARGE{{\bf}Indigestion: Assessing the impact of known and future hash function attacks}

\vspace{1 in}
\large{
\begin{tabular}{c}
Eric Rescorla \vspace{-.1 in}\\
Network Resonance \vspace{-.1 in}\\
\texttt{ekr@networkresonance.com} 
\end{tabular}
}

\end{center}
\end{slide}

\begin{slide}
\heading{Overview}

\begin{itemize}
\item Hash functions are used all over cryptographic protocols
\item Only a few common functions
\begin{itemize}
\item MD5, SHA-1, SHA-256, SHA-384, SHA-512
\item Only MD5 and SHA-1 are commonly used
\item No good theoretical foundation for constructions
\end{itemize}
\item First good attacks on MD5 and SHA-1 published in 2004-2005
\item Impact of current attacks on our protocols?
\item Impact of plausible future attacks?
\item What's being done about it?
\end{itemize}

\end{slide}


\begin{slide}
\heading{Review of hash function terminology}

\begin{description}
\item[Collision] Find $M$, $M'$ st $H(M)=H(M')$
\item[1st preimage] Given $X$, find $M$ st $H(M) = X$
\item[2nd preimage] Given $M$, find $M'$ st $H(M')=H(M)$
\end{description}
\vspace{.15 in}
In a perfect hash function of length $l$:
\begin{itemize}
\item Collisions require $2^{l/2}$ effort to find 
\item 1st and 2nd preimages require $2^l$ effort to find
\end{itemize}
\end{slide}

\begin{slide}
\heading{How hash functions are used}

\begin{itemize}
\item Digital signatures
\item Key derivation (PRFs, KDFs)
\item MACs
\item Data fingerprinting (e.g., Tripwire)
\end{itemize}

\end{slide}


\begin{slide}
\heading{The Current Situation}

\vspace{-.15 in}

\begin{description}
\item[MD5] Collisions can be easily found [WY04][Klima06]``Target'' collisions
can be found with $2^{52}$ effort [SLW06]. 
\item[SHA-1] Collisions in SHA-1 with $2^{63}$ effort (design goal = $80$ bits) [Wang et al.]
\item[Certificates] Lenstra et al. demonstrate a pair of certificates with
different public keys but the same hash (and hence signature) [LWW05]
\item[HMAC] Semi-theoretical forgery attacks [KBBH06][CY06]
\end{description}
\vspace{.05 in}
Important limitations:
\begin{itemize}
\item None of these attacks allows you to compute a preimage
\item The colliders are not totally controllable
\end{itemize}
\end{slide}




\begin{slide}
\heading{Basic Collision Attack}

\begin{itemize}
\item Start with an arbitrary hash state (IV)
\item Compute four blocks $M_0, M_1, M_0', M_1'$ st.
\begin{itemize}
\item  $H(M_0, M_1) = H(M_0', M_1')$
\end{itemize}
\item Relationships between colliders are not free
\begin{itemize}
\item Some bits must be the same, some different, some arbitrary, some fixed
\end{itemize}
\item IV must be known
\item How practical is this?
\begin{itemize}
\item Very for MD5: Klima shows 17 seconds on a 3.2 GHz P4
\item Not so for SHA-1: $2^{63}$ operations
\begin{itemize}
\item No collision published yet for SHA-1
\end{itemize}
\end{itemize}
\end{itemize}
\end{slide}

\begin{slide}
\heading{Target Collision Attack}

\begin{itemize}
\item Start with two messages $M_a, M_b$
\item Compute two suffixes $S_a, S_b$ st.
\begin{itemize}
\item $H(M_a || S_a) = H(M_b || S_b)$
\end{itemize}
\item How practical is this?
\begin{itemize}
\item Suffixes are long (~4000 bits)
\begin{itemize}
\item And random-looking
\end{itemize}
\item Work factor is about $2^{52}$ (for now)
\end{itemize}
\end{itemize}
\end{slide}


\begin{slide}

\heading{Overview}

\begin{itemize}
\item Hash functions are used all over cryptographic protocols
\item Only a few common functions
\begin{itemize}
\item MD5, SHA-1, SHA-256, SHA-384, SHA-512
\item Only MD5 and SHA-1 are commonly used
\item No good theoretical foundation for constructions
\end{itemize}
\item First good attacks on MD5 and SHA-1 published in 2004-2005
\item {\color{red}Impact of current attacks on our protocols?}
\item Impact of plausible future attacks?
\item What's being done about it?
\end{itemize}

\end{slide}


\begin{slide}
\heading{Essential Elements of a Collision Attack}

\begin{enumerate}
\item Need a three party situation
\begin{itemize}
\item Alice generates a message
\item Bob (and maybe Alice) signs it
\item Carol acts upon it
\end{itemize}
\end{enumerate}

\end{slide}


\begin{slide}
\heading{Proposed Uses of Collision Attacks}
\begin{itemize}
\item Contract signing (classic example)
\item Forging certificates
\item Compromised software distribution [Kaminsky]
\item Fake authorizations [Daum and Lucks]
\end{itemize}
\end{slide}


\begin{slide}
\heading{Contract Signing and Collisions}

\vspace{-.4 in}

\begin{itemize}
\item Attacker generates two variants
\begin{itemize}
\item M1 = "I will pay Eric \$10.00/hr"  (a bargain)
\item M2 = "I will pay Eric \$10000/hr"  (a rip-off)
\end{itemize}
\item Attacker gets victim to sign M1 (E.g., in an S/MIME message)
\item Then claims victim signed M2
\begin{itemize}
\item And he has evidence to prove it
\end{itemize}
\item Small problems
\begin{itemize}
\item  Real contracts don't have random garbage
\item Victim has both variants
\end{itemize}
\item Big problem
\begin{itemize}
\item This isn't how contracts actually work
\end{itemize}
\end{itemize}
\end{slide}



\begin{slide}
\heading{Contracts in the Real World}

\vspace{-.2 in}
\begin{itemize}
\item You and I negotiate a contract
\begin{itemize}
\item Your lawyer sends me the final copy
\item I sign the last page
\item I fax it over to you
\item You fax it back
\end{itemize}
\item No attempt is made to bind contents to signature
\begin{itemize}
\item At most, I might initial each page
\item But sometimes, just last page is exchanged!
\end{itemize}
\item Signature is unverified
\begin{itemize}
\item How does relying party know, anyway?
\item An "X" can be binding!
\end{itemize}
\item {\color{red}It's the intention that counts}
\end{itemize}
\end{slide}


\begin{slide}
\heading{Essential Elements of a Collision Attack (Revised)}

\begin{enumerate}
\item Need a three party situation
\begin{itemize}
\item Alice generates a message
\item Bob (and maybe Alice) signs it
\item Carol acts upon it
\end{itemize}
\item \emph{Signature needs to be mechanically evaluated}
\item \emph{Find somewhere to hide the random junk}
\end{enumerate}

\end{slide}


\begin{slide}
\heading{Software Distribution}

\begin{itemize}
\item Many software distributions use hashes for integrity checking
\item Generate a patch for a Foolix 1.0 in two versions
\begin{itemize}
\item One innocuous
\item One containing a remote exploit
\end{itemize}
\item When Foolix 1.1 is released I swing into action
\begin{itemize}
\item Break into Foolix's distribution server
\item Replace the good version with my version
\item ...
\item Profit
\end{itemize}
\item That sounded a lot better in my head...
\end{itemize}

\end{slide}


\begin{slide}
\heading{Problems with Software Distribution Attacks}

\vspace{-.3 in}
\begin{itemize}
\item Hard to predict contents of binaries (prefix)
\begin{itemize}
\item Compiler version, flags, timestamps, ...
\item Especially if other people are checking in stuff
\end{itemize}
\item Somewhat easier on source, but...
\begin{itemize}
\item Other people still might check in
\item Plus CVS/SVN timestamps, Id values, etc.
\end{itemize}
\item Still need to somehow replace the ``good'' copy of the program
\begin{itemize}
\item Theoretically easier if program is P2P-distributed
\item ... Unless that distribution includes its own checksums
\end{itemize}
\item Why go to all this trouble?
\begin{itemize}
\item Most programs aren't that well reviewed anyway
\item How hard is it to insert vulnerabilities anyway?
\end{itemize}
\end{itemize}

\end{slide}

\begin{slide}
\heading{Impact of a Single Collision}
\vspace{-.45 in}
\begin{itemize}
\item Collisions are easy to generate with MD5
\item But expensive with SHA-1 ($2^{63}$)
\begin{itemize}
\item ... none have been generated yet 
\item Eventually one will be as a demonstration
\end{itemize}
\item What can you do then?
\begin{itemize}
\item Use it as a binary switch
\item Daum and Lucks show a switch-hitting
postscript doc
\end{itemize}
\end{itemize}

\begin{tabular}{p{1 in}|p{1 in}}
{\color{blue}\texttt{collider1}} & {\color{red}\texttt{collider2}} \\
\texttt{if(collider1)} & \texttt{if(collider1)} \\
\texttt{  display X} & \texttt{  display X}\\
\texttt{else} & \texttt{else}\\
\texttt{  display Y} & \texttt{  display Y}\\
\end{tabular}
\end{slide}


\begin{slide}

\heading{Target Collisions and Certificates}

\vspace{-.2 in}
\begin{itemize}
\item Basic collision attacks seem hard to mount on certificates
\begin{itemize}
\item Cert is packed too tightly to tweak DNs
\item Best attack is to demonstrate two certs
with same DN and different public keys
\end{itemize}
\item Targeted collisions are better
\begin{itemize}
\item Can start with different DNs
\item And then ``repair'' to a collision
\begin{itemize}
\item Repair blocks hidden in public keys
\end{itemize}
\end{itemize}
\item This isn't currently practical
\begin{itemize}
\item Requires way too many blocks
\item Too many computations
\item And knowledge of prefix values
\end{itemize}
\end{itemize}
\end{slide}



\begin{slide}
\heading{Bottom Line: Current Status}

\begin{itemize}
\item Not affected (much)
\begin{itemize}
  \item Key derivation functions (PRFs)
  \item Peer authentication without non-repudiation (SSL, IPsec,
     SSH, etc.)
  \item Message authentication (HMAC)
  \item Challenge-response protocols (probably)
\end{itemize}
\item� Affected
\begin{itemize}
  \item Non-repudiation (at least technically)
  \item Certificate issuance -- but only in some special cases
  \item Timestamps (maybe)
\end{itemize}
\item {\color{red} This assumes certificates remain secure}
\end{itemize}
\end{slide}


\begin{slide}
\heading{Overview}

\begin{itemize}
\item Hash functions are used all over cryptographic protocols
\item Only a few common functions
\begin{itemize}
\item MD5, SHA-1, SHA-256, SHA-384, SHA-512
\item Only MD5 and SHA-1 are commonly used
\item No good theoretical foundation for constructions
\end{itemize}
\item First good attacks on MD5 and SHA-1 published in 2004-2005
\item Impact of current attacks on our protocols?
\item {\color{red}Impact of plausible future attacks?}
\item What's being done about it?
\end{itemize}

\end{slide}

\begin{slide}
\heading{Potential Future Attacks}

\begin{itemize}
\item Controllable collisions
\item 2nd preimage
\item Compromise of HMAC
\begin{itemize}
\item Forgery
\item Some kind of reversal
\end{itemize}
\end{itemize}
\end{slide}


\begin{slide}
\heading{More Controllable Collisions and Certificates}

\begin{itemize}
\item What if we could really control collisions?
\item Attacker generates two names
\begin{itemize}
\item    Good: \texttt{www.attacker.com}
\item    Bad: \texttt{www.a-victim.com}
\end{itemize}
\item Sends a CSR with good name to CA
\begin{itemize}
\item CA signs cert
\item Attacker now has cert with victim's name
\end{itemize}
\item Two problems
\begin{itemize}
\item Can you predict the prefix?
\item What about the random padding?
\end{itemize}
\end{itemize}
\end{slide}


\begin{slide}
\heading{The Structure of Certificates}

\begin{verbatim}
 TBSCertificate ::= SEQUENCE {
         version               Integer value=2
         serialNumber          Integer (chosen by CA)
         signature             algorithm identifier
         issuer                CA's name
         validity              date range
         subject               subject's name
         subjectPublicKeyInfo  public key
         extensions            arbitrary stuff
 }
\end{verbatim}

\begin{itemize}
\item The signature is over $H(TBSCertificate)$
\end{itemize}

\end{slide}



\begin{slide}
\heading{Prefix Prediction}
\vspace{-.35 in}
\begin{itemize}
\item Knowing which values to use depends on the prefix
\begin{itemize}
\item But the prefix isn't totally fixed
\item This is a total design accident!
\end{itemize}
\item All but serial number and validity are fixed
\begin{itemize}
\item Sequential serial numbers are easy to predict
\begin{itemize}
\item  At least to within a few
\item  Verisign uses $H(time\_us)$ which is hard to predict
\end{itemize}
\item  How quantum is the validity?
\begin{itemize}
\item  Verisign seems to use a fixed "not before" but a "not after" based
       on the current time
\item  So predictable to within a few hundred seconds?
\end{itemize}
\end{itemize}
\item Attacker is likely to need to try the attack a large number of
times 
\item Randomizing serial number is a simple countermeasure
\end{itemize}

\end{slide}


\begin{slide}
\heading{A Vulnerable Certificate Structure}

\begin{verbatim}
TBSCertificate ::= SEQUENCE {
         version               Integer value=3
         signature             algorithm identifier
         issuer                CA's name
         subject               subject's name
         subjectPublicKeyInfo  public key
         serialNumber          Integer (chosen by CA)
         validity              date range
         extensions            arbitrary stuff
}
\end{verbatim}
\end{slide}


%% \begin{slide}
%% \heading{Hiding the randomness}

%% \begin{itemize}
%% \item Remember, we want a specific target name
%% \begin{itemize}
%% \item E.g. \texttt{www.amazon.com}
%% \item Though we have flexibility in the name we send the CA
%% \end{itemize}
%% \item  Random padding can be concealed in pubkey
%% \begin{itemize}
%% \item Remember, modulus doesn't have to be p*q
%% \item As long as we can factor it
%% \begin{itemize}
%% \item     ... which is likely for a random modulus [Back 04]
%% \end{itemize}
%% \item May also be able to construct a valid modulus
%% \end{itemize}
%% \end{itemize}

%% \end{slide}




%% \begin{slide}
%% \heading{2nd Preimage Attacks}

%% \begin{itemize}
%% \item Attack description:
%% \begin{itemize}
%% \item Given some message M
%% \item Find some message M' st H(M) = H(M')
%% \end{itemize}
%% \item Classic example: message forgery
%% \begin{itemize}
%% \item Start with signed "Good" message
%% \item Transform it into signed "Bad" message
%% \end{itemize}
%% \end{itemize}
%% \end{slide}


\begin{slide}
\heading{2nd Preimages and Certificates}

\begin{itemize}
\item This is really serious
\begin{itemize}
\item Attacker should be able to forge a cert of his choice
\item Validity of all certs with this digest would be questionable
\item Say goodbye to any cert-based protocol
\item No useful countermeasures
\end{itemize}
\item How likely do we think this is with MD5?
\begin{itemize}
\item If so, really bad
\item Lots of valid certificates use MD5!
\end{itemize}
\item SHA-1 comfort level is higher
\end{itemize}
\end{slide}



\begin{slide}
\heading{2nd Preimages and Other Protocols}
\vspace{-.35 in}
\begin{itemize}
\item Remember: three major uses of hashes
\begin{itemize}
\item MACs
\item Key expansion
\item Signatures
\end{itemize}
\item Only signatures are directly threatened
\item But they're commonly used
\begin{itemize}
\item SSH, SSL, IPsec key agreement
\begin{itemize}
\item  Signatures are over nonces
\item  Only works if very fast (need to beat timeouts)
\end{itemize}
\item S/MIME authentication
\end{itemize}
\item {\color{red} And of course all but SSH depend on certificates}
\item So, this is bad...
\end{itemize}
\end{slide}



\begin{slide}
\heading{Compromise of HMAC}

\begin{itemize}
\item HMAC is \emph{the} standard MAC for most protocols
\item Proofs of security [Bellare et al.]
\begin{itemize}
\item But these don't apply if the interior hash function is weak [KBBH06][CY06]
\end{itemize}
\item So, what if we had a good attack?
\begin{itemize}
\item Forge new messages without the key
\item Extract the key?
\end{itemize}
\item This turns out to depend on protocol details
\end{itemize}

\end{slide}


\begin{slide}
\heading{Impact of HMAC Forgery: SSL/TLS}

\begin{itemize}
\item SSL/TLS uses authenticate then encrypt (AtE)
\begin{itemize}
\item Send $E(K_e, Message || HMAC(K_m, Message))$
\end{itemize}
\item Hard to get MAC/Message pairs to work with...
\item Block ciphers
\begin{itemize}
\item Can't re-insert the MAC
\begin{itemize}
\item It gets randomized
\end{itemize}
\item And wouldn't match the data in any case
\end{itemize}
\item Stream ciphers
\begin{itemize}
\item Can reinsert MAC
\item ... but only if you know the plaintext
\item And need to know \emph{entire} plaintext to get a match
\end{itemize}
\end{itemize}
\end{slide}

\begin{slide}
\heading{Impact of HMAC Forgery: IPsec}

\begin{itemize}
\item IPsec uses encrypt then authenticate (EtA)
\begin{itemize}
\item Send $E(K_e, Message) || HMAC(K_m, Ciphertext)$
\end{itemize}
\item Easy to get MAC/Message pairs to work with
\item Easy to do an existential forgery
\begin{itemize}
\item Modify ciphertext and produce matching MAC
\item Works with block or CTR-mode ciphers
\end{itemize}
\item Harder to do a targeted forgery
\begin{itemize}
\item Unless you know the plaintext
\begin{itemize}
\item Or part of it
\end{itemize}
\end{itemize}
\end{itemize}
\end{slide}

\begin{slide}
\heading{Impact of HMAC Key Reversal}

\vspace{-.4 in}

\begin{itemize}
\item Can you extract master key from HMAC-based KDF?
\item Plausible scenarios
\begin{itemize}
\item IPsec (remember, EtA)
\begin{itemize}
\item Extract $K_m$, work backward to SKEYSEED
\end{itemize}
\item SSL/TLS (only in NULL encryption mode)
\begin{itemize}
\item Extract $K_m$, work backward to MS
\end{itemize}
\end{itemize}
\item Impact of master key size?
\begin{itemize}
\item Remember: if $|K| > blocksize$, HMAC uses $H(K)$
\item DH ZZ will be hashed
\item Elliptic curve ZZ may not be
\item Static RSA PMS (TLS) may not be?
\item What about intermediate master secrets?
\end{itemize}
\item Truncation helps here
\end{itemize}

\end{slide}

\begin{slide}
\heading{Overview}

\begin{itemize}
\item Hash functions are used all over cryptographic protocols
\item Only a few common functions
\begin{itemize}
\item MD5, SHA-1, SHA-256, SHA-384, SHA-512
\item Only MD5 and SHA-1 are commonly used
\item No good theoretical foundation for constructions
\end{itemize}
\item First good attacks on MD5 and SHA-1 published in 2004-2005
\item Impact of current attacks on our protocols?
\item Impact of plausible future attacks?
\item {\color{red}What's being done about it?}
\end{itemize}
\end{slide}

\begin{slide}
\heading{Transitioning to New Hash Functions}

\begin{itemize}
\item IETF currently upgrading all protocols
\item Currently available
\begin{itemize}
\item SHA-256, SHA-384, SHA-512
\end{itemize}
\item New but ``API compatible constructions''
\begin{itemize}
\item Just define new code points and drop in
\item Hopefully these will come out of NIST competition
\end{itemize}
\item Incompatible constructions
\begin{itemize}
\item Randomized hashes
\begin{itemize}
\item Where do we put the randomizer?
\end{itemize}
\item ...
\end{itemize}
\end{itemize}
\end{slide}


\begin{slide}
\heading{Transition Strategy Goals}

\begin{itemize}
\item \textbf{Backward compatibility}
\begin{itemize}
\item Switch-hitting can speak to old
\item New can speak to switch-hitting
\end{itemize}
\item \textbf{Newest common version}
\begin{itemize}
\item Two switch-hitting implementations should use the new version
\end{itemize}
\item \textbf{Downgrade protection}
\begin{itemize}
\item Attacker can't force you to use a weaker algorithm
\end{itemize}
\item Don't have to change protocol structure again for newer algorithms
\item Diversity in certificates makes this harder
\end{itemize}
\end{slide}


\begin{slide}
\heading{S/MIME Transition Issues}
\vspace{-.2in}
\begin{itemize}
\item Major problem is first message (if signed)
\item Assume sender has two certificates
\begin{itemize}
\item Otherwise there's no choice
\end{itemize}
\item Why not sign twice (once with each certificate)
\begin{itemize}
\item Some implementations don't like partially verifiable messages
\item S/MIME standard wasn't totally clear here
\begin{itemize}
\item Clarification in RFC 4853 [Housley]
\end{itemize}
\end{itemize}
\item Sender has no information about receiver's capabilities
\begin{itemize}
\item Either certificate is potentially wrong
\item Have to guess based on overall upgrade rate
\end{itemize}
\item Subsequent messages can use capabilities attributes
\end{itemize}
\end{slide}


%% \begin{slide}
%% \heading{Uses of Hash Functions in TLS}

%% \begin{itemize}
%% \item Individually and negotiable
%% \begin{itemize}
%% \item Certificates 
%% \item Per-record MAC
%% \end{itemize}
%% \item MD5 and SHA-1 hardwired
%% \begin{itemize}
%% \item Digitally-signed element (for \textsf{ServerKeyExchange} and \textsf{CertificateVerify})
%% \item KDF
%% \item Finished message
%% \end{itemize}
%% \end{itemize}

%%\end{slide}


\begin{slide}
\heading{TLS Transition Issues (I)}

\begin{itemize}
\item IETF doing TLS 1.2 to fix this
\item Certificate Selection
\begin{itemize}
\item Problem: I have certs signed with different algorithms
\begin{itemize}
\item Need to somehow select one
\item New TLS extension: \texttt{cert\_hash\_types}
\end{itemize}
\end{itemize}
\item Per-record MAC
\begin{itemize}
\item Already part of TLS negotiation
\begin{itemize}
\item New cipher suites being created
\end{itemize}
\end{itemize}
\end{itemize}
\end{slide}

\begin{slide}
\heading{TLS Transition Issues (II): KDF}

\begin{itemize}
\item HMAC-based PRF construction
\begin{itemize}
\item XOR SHA-1 and MD5 values
\item Belt-and-suspenders
\begin{itemize}
\item Oops
\end{itemize}
\end{itemize}
\item Switching to a negotiated PRF
\begin{itemize}
\item Old PRF construction with stronger hashes
\item Alternate PRF constructions (GOST, NIST 800-56)
\end{itemize}
\item This applies to \textsf{Finished} PRF too
\end{itemize}

\end{slide}

%% \begin{slide}
%% \heading{TLS Transition Issues (III): Digitally-signed element}

%% \begin{itemize}
%% \item RSA
%% \begin{itemize}
%% \item Sign a concatenated MD5/SHA of handshake messages
%% \end{itemize}
%% \item DSA/ECC
%% \begin{itemize}
%% \item Sign a SHA-1 hash
%% \end{itemize}
%% \item Replace with a single hash function
%% \begin{itemize}
%% \item Again, how to negotiate this?
%% \end{itemize}
%% \end{itemize}
%% \end{slide}


\begin{slide}
\heading{IPsec Transition Issues}

\begin{itemize}
\item AH/ESP transition smoothly
\item IKE has mostly the same issues as SSL/TLS
\begin{itemize}
\item Which certificates to use if you have more than one?
\item Which hash functions initiator should use in ``revised public key mode'' (IKEv1)
\end{itemize}
\item Heuristics proposed in [Bell06]
\item IETF doesn't plan any changes
\end{itemize}
\end{slide}



\begin{slide}
\heading{Summary}

\begin{itemize}
\item Existing systems aren't that threatened by current attacks
\begin{itemize}
\item But enough to cause widespread protocol redesigns
\end{itemize}
\item At least partly by accident
\item There's a surprising amount of robustness in our protocols
\begin{itemize}
\item But certificates are a big single point of failure
\end{itemize}
\item But future attacks could be extremely bad
\item We're in a race with analytic progress
\end{itemize}
\end{slide}


%TODO: Review Laurie signature paper
%TODO: Two blocks?
%TODO: Comparison of collision/preimage countermeasures

\begin{slide}
\heading{Appendix Material}
\end{slide}

\begin{slide}
\heading{Distributed Hash Tables}

\begin{itemize}
\item DHTs typically based on cryptographic hash functions
\begin{itemize}
\item Values typically stored as $H(lookup\_key)$
\begin{itemize}
\item This isn't security critical 
\item Sometimes $H(Value)$
\end{itemize}
\item Identities are often $H(IP)$
\begin{itemize}
\item This is security critical
\end{itemize}
\item Hashes used for data integrity
\end{itemize}
\item Collisions aren't that useful here
\begin{itemize}
\item You just contaminate your own data
\end{itemize}
\item Preimages let you contaminate other people's data
\end{itemize}
\end{slide}


\begin{slide}
\heading{Identity Choice in DHTs}

\begin{itemize}
\item Node $X$ is responsible for storing lookup keys near $X$
\begin{itemize}
\item An attacker might want to choose their location
\item To control specific data
\end{itemize}
\item It seems like compromised hashes should help here
\begin{itemize}
\item Choose specific identity info st. $H(identity)=X$
\end{itemize}
\item But mostly an optimization
\begin{itemize}
\item Easy to brute force for typical DHT sizes
\item Problem is getting control of the identity space
\end{itemize}
\end{itemize}

\end{slide}


\end{document}